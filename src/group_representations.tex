\chapter{Représentations des groupes finis}

Dans ce chapitre, nous étudierons les représentations des groupes finis.

Les représentations des groupes permettent de caractériser les groupes en
étudiant comment ceux-ci agissent sur les espaces vectoriels normés à travers
les automorphismes.

Nous commencerons par donner des définitions générales, sans distinction sur
l'ordre du groupe et on donnera des propositions toutes aussi générales.
Cependant, les exemples donnés ne dépasseront pas le cas fini car nous ne
souhaitons pas étudier d'autres cas dans ce chapitre.

On continuera en se restreignant aux théorèmes se rapportant aux groupes finis:
ceux-ci fourniront des résultats intéressants.

En passant, on regardera un cas particulier de représentations: les
représentations hilbertiennes complexes. Ces dernières sont souvent utilisées en
physique.

\section{Cas général}

\begin{definition} [Représentation d'un groupe, support, réalisation]
	Soit $(G, ., 1_{G})$ un groupe.

	\textbf{Une représentation de $(G, ., 1_{G})$} est un couple $(E,
	\rho)$ où $E$
	est un espace vectoriel normé sur un corps $\mathbb{K}$ et $\rho$
	est morphisme de groupe

	\begin{equation}
		\GSfunction{\rho}{(G, ., 1_{G})}{(\GSisomorphisme{E}, \circ, Id_{E})}
		\label{definition_representation}
	\end{equation}

	où $(\GSisomorphisme{E}, \circ, Id_{E})$ est le groupe formé de l'ensemble
	des automorphismes linéaires de $E$ muni de la loi de
	composition.

	On dit que $E$ est \textbf{le support} de la
	représentation $(E, \rho)$.

	On appelle \textbf{dimension de la représentation $(E, \rho)$} la dimension
	de l'espace vectoriel $E$.

	L'image de $\rho$, noté $\rho(G)$, est \textbf{une réalisation de $G$ dans
		$E$}.
	C'est également \textbf{un sous-groupe} de $(\GSisomorphisme{(E,
		\GSnormeDef{.}{E})}, \circ, Id_{E})$.
\end{definition}

La définition nous dit donc que

\begin{equation}
	\forall g, g' \in G, \, \rho(g g') = \rho(g) \circ \rho(g')
\end{equation}
\begin{equation}
	\forall g \in G, \, \rho(g)^{-1} = \rho(g^{-1})
\end{equation}
\begin{equation}
	\rho(1_{G}) = Id_{E}
\end{equation}

\begin{remarque}
	\begin{enumerate}
		\item Il n'est pas demandé que l'espace vectoriel $E$ soit normé !
			$\GSisomorphisme{E}$ représente simplement l'ensemble des
			applications linéaires de $E$ dans $E$ bijectives.
		\item Soit $g \in G$. On note souvent $\rho_{g}$ à la place de $\rho(g)$
			pour éviter d'utiliser trop de parenthèses.
		\item Pour connaitre toute l'image de $\rho$, il suffit de connaitre
			$\rho$ sur les générateurs de $G$.
	\end{enumerate}
\end{remarque}

Donnons quelques exemples en en construisant.

\begin{exemple}
	Prenons $\mathbb{K} = \complex$ et $(G, ., 1_{G})$ est un groupe fini.

	On construit la représentation $(E, \rho)$ où $E = \complex$ et
	\begin{equation}
		\rho : (G, ., 1_{G}) \rightarrow \overbrace{(\GSisomorphisme{\complex},
		\circ, Id_{\complex})}^{\isomorphe (\complex^{*}, ._{\complex}, 1_{\complex})} : g \rightarrow 1_{\complex}
	\end{equation}
	c'est-à-dire que $\rho$ est constante.
	Cette représentation, de dimension $1$, est appelée \textbf{représentation
	triviale}.
\end{exemple}

\begin{exemple}
	Soit $(\permutationGroup{3}, \circ, 1_{\permutationGroup{3}})$ le groupe des permutations de l'ensemble à $3$ éléments.
	On sait que $\permutationGroup{3} = \generatedGroup{(1 2), (1 2 3)}$

	Par la remarque précédente, si nous voulons construire une représentation de
	$(\permutationGroup{3}, \circ, 1_{\permutationGroup{3}})$, il suffit de construire une représentation des sous-groupes
	$\generatedGroup{(1 2)}$ et $\generatedGroup{(1 2 3)}$ en conservant les
	relations de commutations existantes entre $(1 2)$ et $(1 2 3)$.

	Définissons
	\begin{equation}
		\rho : (\permutationGroup{3}, \circ, 1_{\permutationGroup{3}})
		\rightarrow \overbrace{(\GSisomorphisme{\complex^{2}}, \circ,
		Id_{\complex^{2}})}^{\isomorphe (\inversibleMatrixSpace{2}{\complex},
		., Id_{2})}
	\end{equation}
		tel que
	\begin{equation}
		\rho(1_{\permutationGroup{3}}) =
		\begin{pmatrix}
			1 & 0 \\
			0 & 1
		\end{pmatrix}
		\quad
		\rho( (1 2) ) =
		\begin{pmatrix}
			0 & 1 \\
			1 & 0
		\end{pmatrix}
		\quad
		\rho( (1 2 3) ) =
		\begin{pmatrix}
			e^{\frac{2i\pi}{3}} 	&	0 \\
			0						&	e^{\frac{4i\pi}{3}}
		\end{pmatrix}
	\end{equation}

	Alors, $(\complex^{2}, \rho)$ est une représentation de dimension $2$ du
	groupe $(\permutationGroup{3}, \circ, 1_{\permutationGroup{3}})$.
\end{exemple}

Donnons aussi un lemme qui sera utilisé prochainement.

\begin{lemma}
	Soient $(G, ., 1_{G})$ un groupe et $(E, \rho)$ une représentation de $(G, ., 1_{G})$.

	Alors, pour tout $g \in G$,
	\begin{align}
		\sum_{h \in G} \rho(gh) & = \sum_{h \in G} \rho(hg) \\
		& = \sum_{g' \in G} \rho(g')
	\end{align}
\end{lemma}

\ifdefined\outputproof
\begin{proof}

\end{proof}
\fi

\begin{definition} [Représentation induite]
	Soient $(G, ., 1_{G})$ un groupe et $(E, \rho)$ une représentation de $(G, ., 1_{G})$.

	Soit $(H, ., 1_{G})$ un sous groupe de $(G, ., 1_{G})$.

	Alors, la représentation $(E, \rho)$ de $(G, ., 1_{G})$ induit
	une représentation $(E,
	\rho_{|H})$ de $(H, ., 1_{G})$ et $(E, \rho_{|H})$ est appelée
	\textbf{représentation induite} de $(E, \rho)$ sur $(H,
	., 1_{G})$.
\end{definition}

\begin{definition} [Représentation fidèle/dégénérée]
	Soient $(G, ., 1_{G})$ un groupe et $(E, \rho)$ une représentation de $(G, ., 1_{G})$.

	On dit que $(E, \rho)$ est \textbf{une représentation fidèle} si $\rho$ est
	un morphisme injectif (ie $\ker(\rho) = \GSset{1_{G}}$). Sinon, $(E,
	\GSnormeDef{.}{E}, \rho)$
	est \textbf{une représentation dégénérée}.
\end{definition}

Pour nous familiariser avec la définition de représentation de groupe,
commençons par démontrer une équivalence.

\begin{proposition}
	Soient $(G, ., 1_{G})$ un groupe et $(E, \rho)$ une représentation de $(G, ., 1_{G})$.

	Alors les assertions suivantes sont équivalentes:
	\begin{enumerate}
		\item $(E, \rho)$ est dégénérée.
		\item il existe un sous groupe normal non trivial $(H, ., 1_{G})$ de $(G, ., 1_{G})$
			tel que la représentation induite $(E, \rho_{|H})$ est fidèle.
	\end{enumerate}
\end{proposition}

\ifdefined\outputproof
\begin{proof}

\end{proof}
\fi

\begin{corollary}
	Soit $(G, ., 1_{G})$ un groupe simple. Alors toute représentation $(E,
	\rho)$ de
	$(G, ., 1_{G})$ non nulle est fidèle.
\end{corollary}

\ifdefined\outputproof
\begin{proof}

\end{proof}
\fi

Donnons maintenant une méthode pour construire de nouvelle représentation à
partir d'une connue.
La proposition suivante construit, grace aux classes d'équivalence provenant
d'un quotient d'un groupe sur un de ses groupes normaux, une nouvelle
représentation.

Cette construction consiste à donner la même valeur pour chaque élément se
trouvant dans la même classe d'équivalence.

\begin{proposition}
	\label{prop:build_new_repr_with_equiv_class}
	Soit $(G, ., 1_{G})$ un groupe et soit $(H, ., 1_{H})$ un sous-groupe normal
	de $(G, ., 1_{G})$.

	Soit $(G/H, ., H)$ le groupe quotient de $(G, ., 1_{G})$ sur $(H, .,
	1_{G})$.

	Soit $(E, \rho)$ une représentation de $(G, ., 1_{G})$.

	Posons
	\begin{equation}
		\tau : G \rightarrow \GSisomorphisme{E}
	\end{equation}
	tel que
	\begin{equation}
		\tau(g) = \rho(\tilde{g})
	\end{equation}
	où $\tilde{g} \in G$ tel que $[g]_{G/H} = [\tilde{g}]_{G/H}$.

	Alors, $(E, \tau)$ est une représentation de $(G, .,
	1_{G})$.
\end{proposition}

\ifdefined\outputproof
\begin{proof}

\end{proof}
\fi

Il faut remarquer quelque chose de très essentiel dans la proposition
précédente: plus il y a de sous-groupe normal du groupe initial, plus il y a de
représentation. Cette remarque va jouer un rôle crucial quand nous étudierons
les représentations irréductibles et inéquivalentes d'un groupe fini.


\begin{definition} [Représentation finie/infinie]
	Soient $(G, ., 1_{G})$ un groupe et $(E, \rho)$ une représentation de $(G, ., 1_{G})$.

	On dit que $(E, \rho)$ est \textbf{une représentation finie} si $E$ est de dimension finie.
	Si $E$ n'est pas de dimension finie, on dit que $(E, \rho)$ est \textbf{une
	représentation infinie}.
\end{definition}

Dans le cas d'une représentation finie $(E, \rho)$, on a
\begin{equation}
	(\GSisomorphisme{E}, \circ, Id_{E})
	\isomorphe (\inversibleMatrixSpace{n}{\mathbb{K}}, ., Id_{n})
\end{equation}
où
$\inversibleMatrixSpace{n}{\mathbb{K}}$ est l'ensemble des matrices carrées $n
\cartprod n$ de déterminant non nul et $n = dim(E)$. Les éléments $\rho(g)$ peuvent alors être
vus comme des matrices, ce que nous avons fait dans l'exemple sur
$(\permutationGroup{3}, \circ, 1_{\permutationGroup{3}})$.

En physique, on s'intéresse particulièrement aux représentations finies, et le
corps de base est $\complex$.

Restons dans le cas fini, et prenons une base $(e_{i})_{1 \leq i \leq n}$. On
définit les éléments de matrices $\rho_{ij}(g)$ par

\begin{equation}
	\rho(g)(e_{i}) = \rho(g)_{ij} e_{j} = \rho(g)_{i}^{j} e_{j}
\end{equation}

\subsection{Sous-espace invariant, représentation irréductible et somme directe
de représentations}

On arrive à une classe importante de réprésentation: les représentations
irréductibles. Cette classe nous permettra de déduire des résultats sur les
groupes finis quand nous étudierons les réprésentations à support hilbertien.

Nous verrons qu'étudier cette unique classe permet d'étudier l'entièreté des
représentations d'un groupe. En quelque sorte, elle 'génère' l'ensemble des
représentations.
Pour cela, on étudiera les sommes directes de représentations.

\begin{definition} [Sous espace invariant]
	Soient $(G, ., 1_{G})$ un groupe et $(E, \rho)$ une représentation de $(G, ., 1_{G})$.
	Soit $V$ un sous-espace vectoriel de $E$.

	On dit que $V$ est \textbf{invariant pour $\rho$} si
	\begin{equation}
		\forall g \in G, ., 1_{G}, \rho_{g}(V) \subseteq V
		\label{definition_invariant_subspace}
	\end{equation}
\end{definition}

\begin{proposition}
	$\left\{ O_{E} \right\}$ et $E$ sont des espaces invariants. On les appelle
	les espaces invariants \textbf{triviaux}.
\end{proposition}

\ifdefined\outputproof
\begin{proof}

\end{proof}
\fi

\begin{proposition}
	Soient $(G, ., 1_{G})$ un groupe et $(E, \rho)$ une représentation de $(G, ., 1_{G})$.

	Soit $V$ un sous-espace vectoriel de $E$ invariant sous $\rho$.

	Alors l'application
	\begin{equation}
		\rho^{V} : (G, ., 1_{G}) \rightarrow
		(\GSisomorphisme{(V, \GSnormeDef{.}{V})}, \circ, Id_{V})
	\end{equation}
	tel que
	\begin{equation}
		\rho^{V}(g) = \rho(g)_{|V}
	\end{equation}
	et où $\GSnormeDef{.}{V}$ est la norme $\GSnormeDef{.}{E}$ restreinte à $V$,
	est un morphisme de groupe de $(G, ., 1_{G})$ dans $(\GSisomorphisme{(V,
		\GSnormeDef{.}{V})}, \circ, Id_{V})$.
\end{proposition}

\ifdefined\outputproof
\begin{proof}

\end{proof}
\fi

\begin{definition} [Sous-représentation]
	Soient $(G, ., 1_{G})$ un groupe et $(E, \rho)$ une représentation de $(G, ., 1_{G})$.

	Soient $V$ un sous-espace vectoriel de $E$ invariant sous $\rho$, $\rho^{V}$
	le morphisme défini précédemment.

	Alors $(V, \rho^{V})$ est une représentation de $(G, .,
	1_{G})$ et est appelée \textbf{sous-représentation de $(E, \rho)$}.
\end{definition}

\begin{definition} [Représentation irréductible]
	Soient $(G, ., 1_{G})$ un groupe et $(E, \rho)$ une représentation de $(G, ., 1_{G})$.

	On dit que $(E, \rho)$ est \textbf{une représentation
	irréductible} s'il
	n'existe pas de sous-espaces invariants non-triviaux pour $\rho$. S'il
	existe un sous-espace invariant non-trivial, on dit que $(E, \rho)$ est
	\textbf{une représentation réductible}.
\end{definition}

\begin{remarque}
	Soient $(G, ., 1_{G})$ un groupe et $E$ un espace
	vectoriel normé.

	Toute représentation $(E, \rho)$ de $(G, ., 1_{G})$ de dimension $1$ est
	irréductible.
\end{remarque}

Donnons une caractérisation des représentations irréductibles des groupes finis.

\begin{proposition}
	\label{prop:irreductible_representation_finite_group_is_finite}
	Soient $(G, ., 1_{G})$ un groupe \textit{fini}.

	Toute représentation irréductible non nulle $(E, \rho)$ de $(G, ., 1_{G})$ est finie.

	En d'autres termes, toute représentation irréductible d'un groupe fini est finie.
\end{proposition}

\ifdefined\outputproof
\begin{proof}
	Soit $x \in E$ tel qu'il existe $g \in G$, $\rho_{g}(x) \neq 0_{E}$.
	Posons $\ordergroup{G} = n$ et $G = \GSset{g_{1}, \cdots, g_{n}}$. Prenons
	l'espace vectoriel de $E$ engendré par $\GSset{\rho_{g_{1}}(x) \cdots,
	\rho_{g_{n}}(x)}$ et notons le $V$. Remarquons que $V$ est dimension finie
	et $V \neq \GSset{0_{E}}$
	car il existe $i \in \GSset{1, \cdots, n}$ tel que $\rho_{g_{i}}(x) \neq
	0_{E}$. Montrons maintenant que $V$ est invariant
	sous $G$.

	Soit $g \in G$. Soit $y \in V$ tel que
	\begin{equation}
		y = \sum_{i = 1}^{n} \overbrace{a_{i}}^{\in \mathbb{K}}
		\underbrace{\rho_{g_{i}}(x)}_{\in V}
	\end{equation}

	On a
	\begin{align}
		\rho_{g}(y) & = \rho_{g}(\sum_{i = 1}^{n} a_{i} \rho_{g_{i}}(x)) \\
		& = \sum_{i = 1}^{n} a_{i} \rho_{g}(\rho_{g_{i}}(x)) \\
		& = \sum_{i = 1}^{n} a_{i} (\rho_{g} \circ \rho_{g_{i}})(x) \\
		& = \sum _{i = 1}^{n} a_{i} \rho(g . g_{i})(x) \text{ (car $\rho$ est un
		morphisme)}\\
		& = \sum_{i = 1}^{n} a_{i} \underbrace{\rho(h_{i})(x)}_{\in V}
	\end{align}

	La dernière ligne utilise la définition de $V$.

	Donc $V$ est bien invariant sous $\rho$.

	Comme $(E, \rho)$ est une représentation irréductible, $V = \GSset{0_{E}}$
	ou $V = E$. Comme $V$ est non nul, $V = E$.

	On en déduit que $E$ est de dimension finie. D'où, $(E,
	\rho)$ est une représentation finie.
\end{proof}
\fi

Nous arrivons à la somme directe de représentations: un concept très important.

Prenons deux espaces vectoriels $E_{1}$ et $E_{2}$ et construisons le produit
cartésien $E_{1} \cartprod E_{2}$ de $E_{1}$ et $E_{2}$.

Sur cet ensemble, il est alors possible de poser une structure d'espace
vectoriel, dépendant des structures $E_{1}$ et $E_{2}$: celle-ci se fait
composante par composante. Cet espace vectoriel est noté $E_{1} \oplus E_{2}$
et est appelé \textbf{somme directe de $E_{1}$ et $E_{2}$}.

\begin{definition} [Somme directe de représentations]
	Soit $(G, ., 1_{G})$ un groupe et soient $(E_{1}, \rho_{1})$
	et $(E_{2}, \rho_{2})$
	deux représentations de $(G, ., 1_{G})$.

	On définit le morphisme de groupe \footnote{Le lecteur est invité à
	démontrer que ç'en est un.}
	\begin{equation}
		\rho_{1} \oplus \rho_{2} : (G, ., 1_{G}) \rightarrow (\GSisomorphisme{E_{1}
		\oplus E_{2}}, \circ, Id_{E_{1} \oplus E_{2}})
	\end{equation}
	tel que, pour tout $g \in G$ et pour tout $(x_{1}, x_{2}) \in E_{1} \cartprod
	E_{2}$
	\begin{equation}
		(\rho_{1} \oplus \rho_{2})(g)(\overbrace{x_{1} + x_{2}}^{\in E_{1} \oplus E_{2}}) =
		\underbrace{(\overbrace{\rho_{1}(g) (x_{1})}^{\in E_{1}},
		\overbrace{\rho_{2}(g)(x_{2})}^{\in E_{2}})}_{\in E_{1} \oplus E_{2}}
	\end{equation}.

	On obtient alors une représentations $(E_{1} \oplus E_{2},
	\GSnormeDef{.}{\oplus}, \rho_{1} \oplus
	\rho_{2})$ de $(G, ., 1_{G})$. On dit que $(E_{1} \oplus E_{2}, \rho_{1} \oplus \rho_{2})$ est \textbf{la somme
		directe de $(E_{1}, \rho_{1})$ et $(E_{2},
		\rho_{2})$}.

	De manière générale, on construit la représentation
	\begin{equation}
		(E_{1} \oplus \cdots \oplus E_{m}, \rho_{1} \oplus \cdots \oplus \rho_{m})
	\end{equation}
\end{definition}

Dans le cas où $(E_{1}, \rho_{1})$ et $(E_{2}, \rho_{2})$ sont des
représentations finies, on peut se représenter $\rho_{1} \oplus \rho_{2}$
matriciellement, où pour tout $g \in G$, on a
\begin{equation}
	(\rho_{1} \oplus \rho_{2})(g) =
	\begin{pmatrix}
		\rho_{1}(g)	& 0 \\
		0			& \rho_{2}(g)
	\end{pmatrix}
\end{equation}

Prenons le cas où $E_{1} = E_{2}$ (notons alors $E$ l'espace vectoriel),
$\GSnormeDef{.}{1} = \GSnormeDef{.}{2}$ (notons alors $\GSnorme{.}$ la norme sur
$E$) et $\rho_{1} = \rho_{2}$ (notons alors $\rho$ le morphisme).

On construit alors la représentation $(E \oplus E, \rho
\oplus \rho)$.
La notation $\rho \oplus \rho$ est alors simplifié en $2 \rho$ et dans le cas
des représentations finies, on se retrouve avec la représentation matricielle
suivante

\begin{equation}
	(2 \rho)(g) =
	\begin{pmatrix}
		\rho(g)		& 0 \\
		0			& \rho(g)
	\end{pmatrix}
\end{equation}

Si on travaille avec $n$ identiques espaces vectoriels
(resp. normes et morphismes), on note la somme directe $\rho \oplus \cdots
\oplus \rho$ par $n \rho$.

De manière générale, prenons le cas où le support $E$ est

\begin{equation}
	\overbrace{E_{1} \oplus \cdots E_{1}}^{m_{1} \text{ fois}} \oplus \cdots
	\oplus \overbrace{E_{n} \oplus \cdots \oplus E_{n}}^{m_{n} \text{ fois}}
\end{equation}

et que nous avons les représentations $(E_{1}, \rho_{1}),
\cdots, (E_{n}, \rho_{n})$.

On note alors la représentation

\begin{equation}
	\overbrace{\rho_{1} \oplus \cdots \rho_{1}}^{m_{1} \text{ fois}} \oplus \cdots
	\oplus \overbrace{\rho_{n} \oplus \cdots \oplus \rho_{n}}^{m_{n} \text{ fois}}
\end{equation}

par

\begin{equation}
	\oplus_{i = 1}^{n} m_{i} \rho_{i}
\end{equation}

Dans le cas fini, la représentation matricielle de $\rho := \oplus_{i = 1}^{n}
m_{i} \rho_{i}$ est

\begin{equation}
	(\oplus_{i = 1}^{n} \rho_{i})(g) =
	\begin{pmatrix}
		\rho_{1}(g)		& 0			& 0			&	0			& 0				&
		0 & 0 \\
		0				& \cdots 	& 0				&	0		& 0				&
		0 & 0 \\
		0				& 0 		& \rho_{1}(g)	&	0		& 0				&
		0 & 0 \\
		0				& 0			& 0				&	\cdots	& 0				&
		0				& 0 \\
		0				& 0			& 0				&	0		& \rho_{n}(g)	&
		0				&	0 \\
		0				& 0			& 0				&	0		& 0				&
		\cdots 			& 0 \\
		0				& 0			& 0				&	0		& 0				&
		0				& \rho_{n}(g) \\
	\end{pmatrix}
\end{equation}

Nous verrons que cette représentation est très importante dans la section sur la
représentation régulière.

\begin{definition}
	Soit $(G, ., 1_{G})$ un groupe et soit $(E, \rho)$
	une représentation de $(G, ., 1_{G})$.

	On dit que $(E, \rho)$ est \textbf{complètement
	réductible ou décomposable} s'il existe des représentations irréductibles $(V_{1},
	\rho_{1}), \cdots, (V_{m}, \rho_{m})$ tel que
	\begin{equation}
		\rho = \rho_{1} \oplus \cdots \oplus \rho_{m}
	\end{equation}
	et
	\begin{equation}
		E = V_{1} \oplus \cdots \oplus V_{m}
	\end{equation}
\end{definition}

Le nom de décomposable se rapporte à la représentation matricielle de $\rho$.
En effet, si $E$ est de dimension finie, on peut représenter $\rho$ de la
manière suivante:

\begin{equation}
	\rho(g) =
	\begin{pmatrix}
		\rho_{1}(g)	& 0				&	0			& 0			&	0 \\
		0			& \rho_{2}(g)	&	0			& 0			&	0 \\
		0			& 0				&	\rho_{3}(g)	& 0			&	0 \\
		0			& 0				&	0			& \cdots	&	0 \\
		0			& 0				&	0			& 0			&	\rho_{m}(g) \\
	\end{pmatrix}
\end{equation}

où chaque $\rho_{i}(g)$ est une matrice carrée de taille $dim(V_{i})$.

\subsection{Représentations équivalentes}

% TODO: Donner la motivation des représentations équivalentes.

\begin{definition}
	Soit $(G, ., 1_{G})$ un groupe et soient deux représentations $(E, \rho)$,
	$(F, \tau)$ de $(G, ., 1_{G})$.

	On dit que \textbf{$(E, \rho)$ et $(F, \tau)$ sont (des représentations)
	équivalentes} s'il existe un isomorphisme linéaire
	\begin{equation}
		A : E \rightarrow F
	\end{equation}
	c'est-à-dire
	\begin{equation}
		A \in \GSisomorphisme{E; F}
	\end{equation}
	tel que pour tout $g \in G$,

	\begin{equation}
		\underbrace{\overbrace{A}^{E \rightarrow F} \circ \overbrace{\rho(g)}^{E \rightarrow
		E}}_{E \rightarrow F} = \underbrace{\overbrace{\tau(g)}^{F \rightarrow F} \circ
		\overbrace{A}^{E \rightarrow F}}_{E \rightarrow F}
	\end{equation}
	De manière équivalente, on a
	\begin{equation}
		A \circ \rho(g) \circ A^{-1} = \tau(g)
	\end{equation}

	On peut également dire que $\tau(g)$ est un conjugué de $\rho$.
	On peut donc dire que les représentations équivalentes d'une représentation
	donnée sont ses conjugués.

	Deux représentations qui ne sont pas équivalentes sont dites
	\textbf{inéquivalentes}.
\end{definition}

De cette manière, on \textit{classe} les représentations de $(G, ., 1_{G})$ car la
relation 'être équivalentes' pour deux représentations est une relation
d'équivalence sur l'ensemble des représentations du groupe $(G, ., 1_{G})$. On
parlera de \textbf{classe de représentation}.

\begin{proposition}
	Soit $(G, ., 1_{G})$ un groupe et posons
	\begin{equation}
		\GSrepr{(G, ., 1_{G})}
	\end{equation}
	l'ensemble des représentations de $(G, ., 1_{G})$.

	La relation binaire $\sim$ définie sur $\GSrepr{(G, ., 1_{G})}$ tel que
	\begin{equation}
		(E, \rho) \sim (F, \tau)
	\end{equation}
	si $(E, \rho)$ est une représentation équivalente à $(F, \tau)$
	est une relation d'équivalence.
\end{proposition}

\ifdefined\outputproof
\begin{proof}
	La réflexivité est évidente. La transitivité résulte de la composition des
	applications linéaire continues et la symétrie provient de l'application
	inverse.
\end{proof}
\fi

A partir de maintenant, nous allons nous concentrer sur un représentant de
chaque classe d'équivalences, les propriétés que nous verrons étant stables dans
chaque classe \footnote{Le lecteur sera bien sûr invité à le démontrer.}.

\begin{proposition}
	Soit $(G, ., 1_{G})$ un groupe et soient $(E_{1},
	\rho)$, $(E_{2}, \tau)$ deux représentations de $(G, ., 1_{G})$.

	Soit
	\begin{equation}
		u : E_{1} \rightarrow E_{2}
	\end{equation}
	une application linéaire.

	Posons
	\begin{equation}
		T_{u} : E_{1} \rightarrow E_{2}
	\end{equation}
	tel que
	\begin{equation}
		T_{u}(x) = \frac{1}{\ordergroup{G}} \sum_{g \in G} (\tau(g) \circ u \circ
		\rho(g^{-1})) (x)
	\end{equation}

	Alors,
	\begin{equation}
		\tau \circ T_{u} = T_{u} \circ \rho
	\end{equation}
\end{proposition}

\ifdefined\outputproof
\begin{proof}

\end{proof}
\fi

%\subsection{Sous groupe commutatif}

%Nous avons remarqué, grace à \ref{prop:decomposition_somme_directe}, que le
%simple fait d'étudier les représentations irréductibles nous permettait de
%connaitre toutes les représentations.

%La proposition suivante va nous permettre de donner un critère pour
%éliminer certaines représentations qui ne sont pas irréductibles.

%\begin{proposition}
	%Soient $G$ un groupe, et $H$ un sous groupe abélien.
	%Soit $T \in \GSrepr{G}{V}$.
	%Alors $V$ est de dimension inférieure ou égale à
	%$\frac{\ordergroup{G}}{\ordergroup{H}}$.
%\end{proposition}

Donnons quelques faits qui nous serviront lorsque nous étudierons le lemme de
Shur (\ref{lemma:shur_lemma_1}).

\begin{proposition}
	\label{prop:kernel_and_rank_invariant}
	Soit $(G, ., 1_{G})$ un groupe et soient $(E_{1},
	\rho)$,
	$(E_{2}, \tau)$
	deux représentations de $(G, ., 1_{G})$.

	Soit $\GSfunction{A}{E_{2}}{E_{1}}$ une application linéaire tel
	que
	\begin{equation}
		\forall g \in G, \rho(g) \circ A = A \circ \tau(g)
	\end{equation}

	Alors $\ker{(A)}$ est invariant pour $\tau$ et $Im(A)$ est invariant sous
	$\rho$.
\end{proposition}

\ifdefined\outputproof
\begin{proof}

\end{proof}
\fi

\begin{proposition}
	\label{prop:commute_eigenspace}
	Soit $(G, ., 1_{G})$ un groupe et soit $(E, \rho)$ une
	représentation de $(G, ., 1_{G})$.

	Soit $A : E \rightarrow E$ une application linéaire tel que
	\begin{equation}
		\forall g \in G, \rho(g) \circ A = A \circ \rho(g)
	\end{equation}
	ie $A$ commute avec $\rho(g)$ pour tout $g \in G$.

	Soit $V$ un sous-espace propre de $A$.

	Alors $V$ est sous-espace invariant pour $\rho$.
\end{proposition}

\ifdefined\outputproof
\begin{proof}

\end{proof}
\fi


\subsection{Lemme de Schur}

Nous supposerons maintenant que nous sommes dans le cas $\mathbb{K} = \complex$.

\begin{proposition} [Lemme de Shur 1]
	\label{lemma:shur_lemma_1}
	Soit $(G, ., 1_{G})$ un groupe et soient $(E_{1}, \rho)$,
	$(E_{2}, \tau)$
	deux représentations irréductibles de $(G, ., 1_{G})$.

	Soit $A : E_{2} \rightarrow E_{1}$ une application linéaire tel
	que

	\begin{equation}
		\label{eq:interwining}
		\forall g \in G, \rho(g) \circ A = A \circ \tau(g)
	\end{equation}

	Alors:
	\begin{enumerate}
		\item Si $(E_{1},
	\rho)$ et
	$(E_{2}, \tau)$ sont inéquivalentes,
	alors $A = 0$.

		\item Si $E_{1} = E_{2}$ et $\rho = \tau$, alors $A = \lambda
			Id_{E_{1}}$.
	\end{enumerate}
\end{proposition}

\ifdefined\outputproof
\begin{proof}
	\begin{enumerate}
		\item Supposons que $(E_{1},
			\rho)$,
			$(E_{2}, \tau)$ ne sont pas
			équivalentes. Alors, $A$ n'est pas bijective car il n'existe pas
			d'applications linéaires bijectives répondant à
			l'équation \ref{eq:interwining}. On en déduit que $\ker{(A)} \neq
			\GSset{0_{E_{2}}}$ ou $Im(T) \neq E_{1}$.

			Supposons que $\ker{(A)} \neq \GSset{0_{E_{2}}}$. Par la proposition
			\ref{prop:kernel_and_rank_invariant}, $\ker{(A)}$ est invariant pour $\tau$.
			Comme $\tau$ est irréductible, que $\ker{(A)}$ est invariant pour $\tau$ et
			n'est pas restreint à l'élément nul, $\ker{(A)} = E_{2}$. C'est-à-dire
			que $A = 0$.

			Par un même raisonnement, en utilisant le fait que $Im(A)$ est invariant
			pour $\rho$, on a $A = 0$.

			Donc $A = 0$ dans tous les cas.
		\item Supposons que $E_{1} = E_{2}$ et $\rho = \tau$.
			L'équation \ref{eq:interwining} nous dit que $A$ commute avec
			$\rho(g)$ pour tout $g \in G$. On peut alors utiliser la proposition
			\ref{prop:commute_eigenspace} qui nous dit que tout sous-espace
			propre de $A$ est invariant pour $\rho$.

			Soit $\lambda$ une valeur propre de $A$ (celle-ci existe car nous
			travaillons dans le cas complexe) et posons $E_{\lambda}$ le
			sous-espace propre associé à la valeur propre $\lambda$. Nous avons
			alors $E_{\lambda}$ qui est invariant pour $\rho$, et $E_{\lambda}
			\neq \GSset{0_{E_{1}}}$.

			Comme $\rho$ est irréductible, on a $E_{\lambda} = E_{1}$. Donc,
			$A = \lambda Id_{E_{1}}$.
	\end{enumerate}
\end{proof}
\fi

Nous avons une conséquence direct du \ref{lemma:shur_lemma_1}.

\begin{corollary}
	Soit $(G, ., 1_{G})$ un groupe commutatif et soit $(E_{1},
	\rho)$,
	une représentation irréductible de $(G, ., 1_{G})$.

	Alors $E_{1}$ est de dimension $1$ (donc $E_{1} = \complex$).
\end{corollary}

\ifdefined\outputproof
\begin{proof}

\end{proof}
\fi

Ce corollaire nous montre que les seules représentations irréductibles d'un
groupe commutatif sont de degré 1. C'est-à-dire que lorsque nous étudions les
représentations irréductibles d'un groupe abélien, nous étudions des
sous-groupes (finis) $\complex$.

En particulier, les seuls sous-groupes finis de $\complex$ sont les groupes des
racines de l'unité. Étudier les représentation irréductibles de $G$ sur
$\complex$ revient à étudier les sous-groupes abéliens de $G$, et donc en
particulier les sous-groupes cycliques.

Donnons un autre corollaire du lemme de Shur \ref{lemma:shur_lemma_1}, reposant
sur la définition de la fonction $T_{u}$ donnée précédemment.

\begin{corollary}
	Soit $(G, ., 1_{G})$ un groupe fini et soient $(E_{1},
	\rho)$,
	$(E_{2}, \tau)$
	deux représentations irréductibles de $(G, ., 1_{G})$.

	Soit $u : E_{1} \rightarrow E_{2}$ une application linéaire et
	posons $T_{u} : E_{1}  \rightarrow E_{2}$ tel que
	\begin{equation}
		T_{u}(x) = \frac{1}{\ordergroup{G}} \sum_{g \in G} (\tau(g) \circ u \circ
		\rho(g^{-1})) (x)
	\end{equation}

	Alors
	\begin{enumerate}
		\item Si $(E_{1}, \rho)$,
	$(E_{2}, \tau)$ sont inéquivalentes,
	alors $T_{u} = 0$.
		\item Si $E_{1} = E_{2}$ et $\rho = \tau$, alors $T_{u} = \lambda
			Id_{E_{1}}$ avec $\lambda = \frac{Tr(u)}{\dim{E_{1}}}$.
	\end{enumerate}
\end{corollary}

\ifdefined\outputproof
\begin{proof}
	Il suffit de vérifier que $T_{u}$ répond à l'équation \ref{eq:interwining}
	et d'utiliser le lemme de Shur.
\end{proof}
\fi

Remarquons que ce corollaire demande une hypothèse en plus: le
groupe $(G, ., 1_{G})$ doit être fini !

Remarquons aussi que les hypothèses de finitude sur le groupe $(G, ., 1_{G})$ et
l'irréductibilité des représentations impliquent immédiatement, par
\ref{prop:irreductible_representation_finite_group_is_finite}, que $E_{1}$ et
$E_{2}$ sont de dimensions finies.

% TODO: A reformuler car mal expliqué. Il faudrait voir si en remplaçant E_{1}
% est isomorphe à E_{2} est valable pour le lemme de Shur comme deuxième
% proposition.

%En particulier, tous les cas sont bien envisagés car si $E_{1} \neq E_{2}$, il
%est évident que les représentations ne sont pas équivalentes car $E_{1}$ et
%$E_{2}$ seraient isomorphes en tant qu'espace vectoriel, alors que ceux-ci ne le
%sont pas.

\section{Représentation dans les espaces de Hilbert}

Nous supposerons maintenant que nous sommes dans le cas $\mathbb{K} = \complex$.

Depuis le début, nous nous sommes uniquement intéressés au cas où $(E,
\GSnormeDef{.}{E})$ est un espace vectoriel normé quelconque.

Prenons maintenant le cas où $E$ est un espace de Hilbert,
c'est-à-dire que la norme $\GSnormeDef{.}{E}$ peut-être définie à travers un
produit scalaire. On parle alors de \textbf{représentation hilbertienne}.

Nous utiliserons la notation $(\Hilbert, \dotprod{.}{.})$ à la place de $(E,
\GSnormeDef{.}{E})$, où $\dotprod{.}{.}$ est le produit scalaire induisant la
norme $\GSnormeDef{.}{E}$, pour rester cohérent avec les notations usuelles
utilisées dans la théorie des espaces de Hilbert.

Dans les espaces de Hilbert, on peut définir les opérateurs unitaires qui
sont les opérateurs $U$ tel que $U^{*}$, l'opérateur adjoint, est égale à
$U^{-1}$, c'est-à-dire
\begin{equation}
	U \circ U^{*} = U^{*} \circ U = Id_{\Hilbert}
\end{equation}

%TODO: expliquer l'importance des opérateurs unitaires.

Nous noterons $\GSortho{V}$ l'orthogonal du sous-espace vectoriel $V$, c'est-à-dire
\begin{equation}
	\GSortho{V} = \GSsetDef{x \in \Hilbert}{\forall y \in V, \, \dotprod{x}{y} = 0}
\end{equation}

\begin{definition} [Représentation unitaire]
	Soit $(G, ., 1_{G})$ un groupe et soit $(\Hilbert, \dotprod{.}{.}, \rho)$ une représentation de
	$(G, ., 1_{G})$ où $(\Hilbert, \dotprod{.}{.})$ est un espace de Hilbert.

	On dit que $(\Hilbert, \dotprod{.}{.}, \rho)$ est \textbf{une représentation unitaire} si
	pour tout $g \in G$, $\rho(g)$ est unitaire.
\end{definition}

On montrera que les représentations unitaires jouent un rôle important pour les
représentations des groupes finis. En fait, on montrera que dans chaque classe
d'équivalence où apparait une représentation hilbertienne, il y a une
représentation unitaire.

En d'autres termes,

\begin{proposition}
	Soit $(G, ., 1_{G})$ un groupe \textit{fini}.

	Alors, toute représentation hilbertienne est équivalente à une
	représentation unitaire.
\end{proposition}

\ifdefined\outputproof
\begin{proof}
	Soit $(\Hilbert, \dotprod{.}{.}_{\Hilbert}, \rho)$ une représentation hilbertienne de
	$(G, ., 1_{G})$.
	Nous devons montrer qu'il existe une représentation hilbertienne unitaire
	$(\mathcal{K}, \dotprod{.}{.}_{\mathcal{K}}, \tau)$ équivalente à
	$(\Hilbert, \dotprod{.}{.}_{\Hilbert}, \rho)$.

	Prenons $\mathcal{K} = \Hilbert$ et $\tau = \rho$. Il nous reste à
	construire le produit scalaire sur $\mathcal{K}$ qui permettra d'obtenir ce
	que l'on souhaite.

	Soient $x, y \in \mathcal{K}$. Posons
	\begin{equation}
		\dotprod{x}{y}_{\mathcal{K}} = \frac{1}{\ordergroup{G}} \sum_{g \in G}
		\dotprod{\rho_{g}(x)}{\rho_{g}(y)}_{\Hilbert}
	\end{equation}

	Le produit scalaire $\dotprod{.}{.}_{\mathcal{K}}$ ainsi défini induit une
	représentation hilbertienne unitaire $(\mathcal{K},
	\dotprod{.}{.}_{\mathcal{K}}, \rho)$.

	Cette représentation est bien équivalente à $(\Hilbert,
	\dotprod{.}{.}_{\Hilbert}, \rho)$ en prenant $A = Id_{\Hilbert}$.
\end{proof}
\fi

En d'autres termes, lorsque nous étudions une représentation hilbertienne d'un
groupe fini, on peut
supposer, sans perte de généralité que nous travaillons avec une représentation
unitaire.
Les propositions suivantes préciseront toujours que nous avons une
représentation unitaire si c'est le cas (et bien sûr si nous sommes dans le cas
d'un groupe fini), mais il faudra se souvenir que nous
pouvons généraliser à n'importe quelle représentation hilbertienne.

%%Ajouter la représentation matricielle.

%Donnons quelques représentations décomposables.

%\begin{proposition}
	%Soit $T \in \GSrepr{G}{\Hilbert}$ unitaire qui est réductible. Alors $T$
	%est complètement réductible.
%\end{proposition}

%\ifdefined\outputproof
%\begin{proof}
	%Comme $T$ est réductible, il existe un sous-espace invariant non trivial.
	%Notons le $V$.
	%Nous voulons alors montrer que $\GSortho{V}$ est également invariant pour
	%$T$. On aura alors que $T$ est complètement réductible.
%\end{proof}
%\fi

\begin{remarque}
	Prenons une représentation hilbertienne $(\Hilbert, \dotprod{.}{.}_{\Hilbert}, \rho)$ de
	$(G, ., 1_{G})$ et supposons, sans perte de généralité, que celle-ci est
	unitaire.

	On a alors la relation suivante
	\begin{equation}
		\forall g \in G \, \rho(g^{-1}) = {(\rho(g)^{*})}^{t}
	\end{equation}
	c'est-à-dire que, dans le cas d'une représentation finie, les composantes de
	la matrice représentant $\rho(g^{-1})$
	deviennent
	\begin{equation}
		\rho(g^{-1})_{ij} = \conjuguate{\rho(g)_{ji}}
	\end{equation}
\end{remarque}

\section{Caractères et relations d'orthogonalité}

\subsection{Fonctions sur un groupe}

Prenons un groupe $(G, ., 1_{G})$ et concentrons-nous sur le corps des complexes
$\complex$.

On construit l'ensemble des fonctions de $G$ dans $\complex$, qu'on note
$\complex[G]$. Cet ensemble peut être muni d'une structure d'espace vectoriel
complexe.

On peut définir l'addition interne $+_{\complex[G]}$ tel que, pour tout $f, g
\in \complex[G]$ et pour tout $h \in G$,
\begin{equation}
	(f +_{\complex[G]} g) (\overbrace{h}^{\in G}) = \underbrace{f(g)}_{\in
	\complex} + \underbrace{g(h)}_{\in \complex}
\end{equation}

et la multiplication scalaire $._{\complex[G]}$, tel que pour tout $z \in \complex$,
pour tout $f \in \complex[G]$ et pour tout $h \in G$,
\begin{equation}
	(z ._{\complex[G]} f) (h) = z \overbrace{f(h)}^{\in \complex}
\end{equation}

en défissant l'élément neutre $0_{\complex[G]}$ comme la fonction nulle.

On obient bien un espace vectoriel complexe $(\complex[G], +_{\complex[G]},
	._{\complex[G]}, 0_{\complex[G]})$. \footnote{Cette construction peut être
		faite pour n'importe quel corps. On note alors la structure
		$(\mathbb{K}[G], +_{\mathbb{K}[G]}, ._{\mathbb{K}[G]},
		0_{\mathbb{K}[G]})$.}

		Dans le cas d'un groupe fini\footnote{Le cas compact est possible}, nous pouvons définir un produit scalaire sur
$\complex[G]$. En effet, pour tout $f_{1}, f_{2} \in \complex[G]$, on définit

\begin{equation}
	\dotprod{f_{1}}{f_{2}}_{\complex[G]} = \frac{1}{\ordergroup{G}} \sum_{g \in G}
	\conjuguate{f_{1}(g)} f_{2}(g)
\end{equation}

L'espace $(\complex[G], \dotprod{.}{.}_{\complex[G]})$ alors formé donne un
espace de Hilbert.\footnote{Le lecteur est invité à vérifier que la fonction
	$\dotprod{.}{.}_{\complex[G]}$ est bien un produit scalaire et que l'espace
est alors complet pour la norme induite.}.

Terminons avec une proposition qui nous aidera par la suite.

\begin{proposition}
	Soit $(G, ., 1_{G})$ un groupe fini et soit $(\complex[G], +_{\complex[G]},
	._{\complex[G]}, 0_{\complex[G]})$ l'espace vectoriel des fonctions de $G$
	dans $\complex$.

	Alors $(\complex[G], +_{\complex[G]}, ._{\complex[G]}, 0_{\complex[G]})$ est
	de dimension finie. Plus précisément, sa dimension vaut $\ordergroup{G}$.
\end{proposition}

\ifdefined\outputproof
\begin{proof}
	Soit $g \in G$. Posons la fonction
	\begin{equation}
		e_{g} : G \rightarrow \complex
	\end{equation}
	tel que, pour tout $h \in G$,
	\begin{equation}
		e_{g}(h) = \delta_{g}^{h}
	\end{equation}

	Les fonctions $e_{g}$, $g$ parcourant $G$, sont des fonctions de
	$\complex[G]$, linéairément indépendante.

	De plus, l'ensemble $\GSsetDef{e_{g}}{g \in G}$ génère bien l'ensemble
	$\complex[G]$ car si on prend une fonction $f : G \rightarrow \complex$, on
	a
	\begin{equation}
		f = \sum_{g \in G} \overbrace{f(g)}^{\in \complex} \overbrace{e_{g}}^{\in
		\complex[G]}
	\end{equation}

	L'ensemble $\GSsetDef{e_{g}}{g \in G}$ forment donc une base $\complex[G]$
	et comportent $\ordergroup{G}$ éléments. Cette base est appelé \textbf{base
	usuelle de $\complex[G]$}.
\end{proof}
\fi

Cette construction va nous permettre de travailler avec des fonctions qui vont
de $G$ dans $\complex$, et de tirer des propositions et théorèmes intéressants
sur les représentations de groupe.

\subsection{Caractère}

Nous allons nous concentrer sur les représentations finies des groupes finis.
Dans la suite, $(G, ., 1_{G})$ représente un groupe fini et $(E, \rho)$ une
représentation finie (ie $E$ est de dimension finie).

Rappelons que dans le cas d'une représentation finie, celle-ci peut être
représentée par une matrice carrée $n \cartprod n$ où $n$ est la dimension de
l'espace vectoriel.

\begin{definition}
	Soit $(G, ., 1_{G})$ un groupe fini et soit $(E, \rho)$ une
	représentation finie de $(G, ., 1_{G})$.

	On définit \textbf{le caractère de $(E, \rho)$} comme la
	fonction
	\begin{equation}
		\chi_{\rho} : G \rightarrow \complex
	\end{equation}
	tel que pour tout $g \in G$
	\begin{equation}
		\chi_{\rho}(g) = Tr(\rho_{g}) = \sum_{i = 1}^{\dim(E)} \rho(g)_{ii}
	\end{equation}
\end{definition}

Remarquons que le caractère d'une répresentation d'un groupe est une fonction de
$\complex[G]$. Nous étudierons plus tard le produit scalaire entre deux
caractères.

Donnons quelques premières propriétés sur les caractères des représentations.

\begin{proposition}
	Soit $(G, ., 1_{G})$ un groupe fini et soit $(E, \rho)$ une
	représentation finie de $(G, ., 1_{G})$.

	Soit $\chi_{\rho}$ le caractère de la représentation $(E,
	\rho)$.
	\begin{enumerate}
		\item Pour tout $g \in G$, $\chi_{\rho}(g^{-1}) =
			\conjuguate{\chi_{\rho}(g)}$
		\item $\chi_{\rho}(1_{G}) = \dim(E)$
		\item Si $\rho = \rho_{1} \oplus \rho_{2}$, alors $\chi_{\rho} =
			\chi_{\rho_{1}} + \chi_{\rho_{2}}$.
		\item Soit $h \in G$. Alors, pour tout $g \in G$,
			$\chi_{\rho}(ghg^{-1}) = \chi_{\rho}(h)$. En d'autres termes, le
			caractère est stable dans chaque sous-groupe normal.
	\end{enumerate}
\end{proposition}

\ifdefined\outputproof
\begin{proof}

\end{proof}
\fi

\begin{proposition}
	Le caractère d'une représentation finie d'un groupe fini est constante par
	classe. C'est-à-dire que si nous prenons deux représentations équivalentes,
	leur caractère seront les mêmes.
\end{proposition}

\ifdefined\outputproof
\begin{proof}

\end{proof}
\fi

\subsection{Relations d'orthogonalité}

Nous arrivons aux relations d'orthogonalité sur les \textbf{groupes finis} des
\textbf{représentations finies}.

N'oublions pas que, d'après la proposition
\ref{prop:irreductible_representation_finite_group_is_finite}, une
représentation irréductible d'un groupe fini est toujours une représentation
finie.

Rappelons que le caractère d'une représentation est une fonction appartenant à
l'espace de Hilbert $\complex[G]$. Nous pouvons donc étudier le résultat du
produit scalaire entre deux caractères.

De plus, nous avons donné quelques résultats grace au lemme de Shur. Nous allons
donner des résultats importants qui en découlent.

Commençons d'abord par les relations d'orthogonalité sur les représentations. On
arrivera alors à une relation semblable sur les caractères.

\begin{proposition}
	Soit $(G, ., 1_{G})$ un groupe fini et soient $(E_{1}, \rho)$,
	$(E_{2}, \tau)$
	deux représentations irréductibles de $(G, ., 1_{G})$.

	Alors
	\begin{enumerate}
		\item Si $(E_{1}, \rho)$,
			$(E_{2}, \tau)$ sont
			inéquivalentes, alors
			\begin{equation}
				\label{eq:orthogonality_relation_inequivalente}
				\forall i, j, k, l \, \sum_{g \in G} \tau(g)_{kl}
				\rho(g^{-1})_{ji}
				= 0
			\end{equation}
		\item Si $E_{1} = E_{2}$ et $\rho = \tau$, alors
		\begin{equation}
			\label{eq:orthogonality_relation_equivalente}
			\forall i, j, k, l \, \frac{1}{\ordergroup{G}} \sum_{g \in G}
			\rho(g)_{kl} \rho(g^{-1})_{ji}
			= \frac{1}{\dim{(E)}} \delta_{k}^{i} \delta_{l}^{j}
		\end{equation}
	\end{enumerate}
\end{proposition}

\ifdefined\outputproof
\begin{proof}

\end{proof}
\fi

\begin{remarque}
	Dans le cas hilbertien et si les représentations sont unitaires, les équations
	\ref{eq:orthogonality_relation_inequivalente} et
	\ref{eq:orthogonality_relation_equivalente} deviennent
	\begin{equation}
		\forall i, j, k, l \, \sum_{g \in G} \tau(g)_{kl}
		\conjuguate{\rho(g)_{ij}}
		= 0
	\end{equation}
	et
	\begin{equation}
		\forall i, j, k, l \, \frac{1}{\ordergroup{G}} \sum_{g \in G}
		\rho(g)_{kl} \conjuguate{\rho(g)_{ij}}
		= \frac{1}{\dim{(E)}} \delta_{k}^{i} \delta_{j}^{l}
	\end{equation}
	car
	\begin{equation}
		\rho(g)^{-1} = {(\rho(g)^{*})}^{t}
	\end{equation}
\end{remarque}

\begin{proposition}
	Soit $(G, ., 1_{G})$ un groupe fini et soient $(E_{1}, \rho)$,
	$(E_{2}, \tau)$
	deux représentations finies de $(G, ., 1_{G})$.

	Alors
	\begin{enumerate}
		\item Si $(E_{1}, \rho)$,
		$(E_{2}, \tau)$ sont irréductibles
		et inéquivalentes, alors
		\begin{align}
			\dotprod{\chi_{\rho}}{\chi_{\tau}}_{\complex[G]} & = 0 \\
			& = \frac{1}{\ordergroup{G}} \sum_{g \in G}
			\conjuguate{\chi_{\rho}(g)} \chi_{\tau}(g) \\
			& = \frac{1}{\ordergroup{G}} \sum_{g \in G}
			\chi_{\rho}(g^{-1}) \chi_{\tau}(g)
		\end{align}
		En d'autres termes, les caractères de deux représentations finies irréductibles et
		inéquivalentes d'un groupe fini sont orthogonaux.

		\item Si $(E_{1}, \rho)$ est une
			représentation irréductible, alors
			\begin{align}
				\dotprod{\chi_{\rho}}{\chi_{\rho}}_{\complex[G]} & = 1 \\
				& = \frac{1}{\ordergroup{G}} \sum_{g \in G}
				\conjuguate{\chi_{\rho}(g)} \chi_{\rho}(g) \\
				& = \frac{1}{\ordergroup{G}} \sum_{g \in G}
				\norm{\chi_{\rho}(g)}^{2}
			\end{align}
			En d'autres termes, le caractère de toute représentation finie irréductible d'un
			groupe fini est normé.
	\end{enumerate}
\end{proposition}

\ifdefined\outputproof
\begin{proof}

\end{proof}
\fi

Nous en venons à un théorème important.

\begin{theorem}
	\label{thm:uir_set_orthonormal}
	Soit $(G, ., 1_{G})$ un groupe fini.

	Les caractères des représentations irréductibles inéquivalentes de
	$(G, ., 1_{G})$ forment un ensemble d'éléments orthogonaux de $\complex[G]$.
\end{theorem}

\ifdefined\outputproof
\begin{proof}
	Evident selon la proposition précédente.
\end{proof}
\fi

\begin{corollary}
	\label{corollary:finite_number_iir}
	Soit $(G, ., 1_{G})$ un groupe fini.

	L'ensemble des représentations inéquivalentes et irréductibles de $(G, .,
	1_{G})$ est de cardinalité finie.

	En d'autres termes, il y a un nombre fini de représentations inéquivalentes et irréductibles pour un
	groupe fini donné.
\end{corollary}

\ifdefined\outputproof
\begin{proof}
	On utilise le théorème précédent et le fait que $\complex[G]$ est de
	dimension finie si $(G, ., 1_{G})$ est un groupe fini.
\end{proof}
\fi

Le corollaire \ref{corollary:finite_number_iir} va nous permettre d'utiliser une
notation simplifiée pour représenter les représentations finies d'une groupe
fini.

\begin{notation}
	Soit $(G, ., 1_{G})$ un groupe fini.
	Le corollaire \ref{corollary:finite_number_iir} nous dit qu'il existe un
	nombre fini de représentation irréductibles inéquivalentes de $(G, .,
	1_{G})$.

	Supposons qu'il existe $N$ représentations irréductibles
	inéquivalentes de $(G, ., 1_{G})$. Nous allons noter celles-ci $(E_{1},
	\rho_{1})$, \ldots, $(E_{N}, \rho_{N})$.
\end{notation}

En utilisant cette notation, nous pouvons énoncer les relations d'orthogonalités
de la manière suivante.

Soit $(G, ., 1_{G})$ un groupe fini et soient $(E_{1},
	\rho_{1})$, \ldots, $(E_{N},
	\rho_{N})$ ses représentations irréductibles inéquivalentes.
	On a alors
	\begin{equation}
		\dotprod{\chi_{\rho_{i}}}{\chi_{\rho_{j}}}_{\complex[G]} =
		\delta_{i}^{j}
	\end{equation}

\section{Table de caractères}

Nous avons vu précédemment qu'il existait un nombre fini de représentations
finies hilbertiennes irréductibles et inéquivalentes pour un groupe fini donné.
On les numérote de $1$ à $N$.

De même, pour un groupe donné, il existe un nombre fini de sous-groupe normaux.

De plus, pour une représentation donné d'un groupe donné, le caractère de la
représentation est constante sur chaque sous-groupe normal.

Finalement, pour connaitre les valeurs du caractère d'une représentation, il
suffit de connaitre sa valeur sur un représentation de chaque sous-groupe
normal.

Soit $(G, ., 1_{G})$ un groupe fini possédant $M$ sous-groupes normaux. Notons
les sous-groupes normaux de $(G, ., 1_{G})$ par $(H_{1}, ., 1_{G}), \cdots,
(H_{M}, ., 1_{G})$.

Prenons, pour chaque sous-groupe normal $(H_{j}, ., 1_{G})$ un élément $h_{j}$
qui servira de représentant pour le calcul des caractères de chaque
représentation.

Nous pouvons placer, dans un tableau, chaque représention finie hilbertienne
irréductible inéquivalente $(E_{i}, \rho_{i})$ dans une ligne, et dans chaque colonne, écrire le
représentant $h_{j}$ du sous-groupe normal $(H_{j}, ., 1_{G})$.

A l'intersection de la ligne $i$ et de la colonne $j$ se retrouvera l'évaluation
du caractère $\chi_{\rho_{i}}$ de la représentation $(E_{i},
\rho_{i})$ en $h_{j}$.

On se retrouve alors avec un tableau de $N$ lignes et $M$ colonnes.

\[\setlength\tabcolsep{2pt}\def\arraystretch{1.5}
\begin{array}{c|c|c|c}
	& \ordergroup{H_{1}} & \cdots \cdots & \ordergroup{H_{M}} \\
	& h_{1} & \cdots \cdots & h_{M} \\
	\hline
	\cdots \cdots & \cdots \cdots & \cdots \cdots & \cdots \cdots \\
	\chi_{\rho_{i}} & \chi_{\rho_{i}}(h_{1}) & \cdots \cdots & \chi_{\rho_{i}}(h_{M})
	\\
	\cdots \cdots & \cdots \cdots & \cdots \cdots & \cdots \cdots \\
	\chi_{\rho_{j}} & \chi_{\rho_{j}}(h_{1}) & \cdots \cdots & \chi_{\rho_{j}}(h_{M})
	\\
	\cdots \cdots & \cdots \cdots & \cdots \cdots & \cdots \cdots \\
\end{array}
\]

\begin{proposition}
	Soit $(G, ., 1_{G})$ un groupe fini et soient $(E_{1},
	\rho_{1}), \cdots, (E_{N}, \rho_{N})$ ses représentations
	irréductibles inéquivalentes.

	Soient $(H_{1}, ., 1_{G}), \cdots, (H_{M}, ., 1_{G})$ les sous-groupes
	normaux de $(G, ., 1_{G})$ et $h_{i}$ un représentant du sous-groupe normal
	$(H_{i}, ., 1_{G})$.

	Alors, pour tout $k, j \in \GSset{1, \cdots, N}$, on a
	\begin{equation}
		\dotprod{\chi_{\rho_{k}}}{\chi_{\rho_{j}}}_{\complex[G]} =
		\frac{1}{\ordergroup{G}} \sum_{i = 1}^{M} \ordergroup{H_{i}} \, \,
		\conjuguate{\chi_{\rho_{k}}(h_{i})} \, \, \chi_{\rho_{j}}(h_{i})
	\end{equation}
\end{proposition}

\ifdefined\outputproof
\begin{proof}

\end{proof}
\fi

\begin{theorem}
	Soit $(G, ., 1_{G})$ un groupe fini et soit $(E, \rho)$
	une représentation de $(G, ., 1_{G})$.
	Soit $\chi_{\rho}$ le caractère de la représentation $(E, \rho)$.

	Alors, il existe $(E_{1}, \rho_{1}), \cdots, (E_{n},
	\rho_{n})$ des représentations irréductibles
	inéquivalentes, et des
	naturels non nuls $m_{1}, \cdots, m_{n}$ tel que
	\begin{equation}
		\rho = \oplus_{i = 1}^{n} m_{i} \rho_{i}
	\end{equation}
	et
	\begin{equation}
		m_{i} = \dotprod{\chi_{\rho_{i}}}{\chi_{\rho}}_{\complex[G]}
	\end{equation}
\end{theorem}

\ifdefined\outputproof
\begin{proof}

\end{proof}
\fi

\begin{definition}
	Soit $(G, ., 1_{G})$ un groupe fini et soit $(E, \rho)$
	une représentation de $(G, ., 1_{G})$.

	Soient les représentations irréductibles inéquivalentes $(E_{1},
	\rho_{1}), \cdots, (E_{n},
	\rho_{n})$ et les naturels non nuls $m_{1}, \cdots,
	m_{n}$ tel que
	\begin{equation}
		\rho = \oplus_{i = 1}^{n} m_{i} \rho_{i}
	\end{equation}

	Le naturel $m_{i}$ est \textbf{la multiplicité de $\rho_{i}$ dans $\rho$} et
	le morphisme $m_{i} \rho_{i}$ est appelé \textbf{composante isotopique de
		$\rho_{i}$ dans $\rho$}.
\end{definition}

\begin{corollary}
	Soit $(G, ., 1_{G})$ un groupe fini et soient $(E_{1},
	\rho_{1})$ et $(E_{2}, \rho_{2})$
	deux représentations de $(G, ., 1_{G})$.

	Posons $\chi_{\rho_{1}}$ le caractère de la représentation $(E_{1},
	\rho_{1})$ et $\chi_{\rho_{2}}$ le caractère de la
	représentation $(E_{2}, \rho_{2})$.

	Alors les assertions suivantes sont équivalentes.
	\begin{enumerate}
		\item Les représentations $(E_{1},
			\rho_{1})$ et $(E_{2}, \rho_{2})$ sont équivalentes.
		\item Les représentations $(E_{1},
			\rho_{1})$ et $(E_{2}, \rho_{2})$ ont même caractère, ie
			$\chi_{\rho_{1}} = \chi_{\rho_{2}}$.

	\end{enumerate}
\end{corollary}

\ifdefined\outputproof
\begin{proof}

\end{proof}
\fi

\begin{proposition}
	Soit $(G, ., 1_{G})$ un groupe fini et soit $(E, \rho)$
	une représentation de $(G, ., 1_{G})$.

	Soient les représentations irréductibles inéquivalentes $(E_{1},
	\rho_{1}), \cdots, (E_{n},
	\rho_{n})$ et les naturels non nuls $m_{1}, \cdots,
	m_{n}$ tel que
	\begin{equation}
		\rho = \oplus_{i = 1}^{n} m_{i} \rho_{i}
	\end{equation}

	Soit $\chi_{\rho}$ le caractère de la représentation $(E,
	\rho)$.

	Alors
	\begin{equation}
		\dotprod{\chi_{\rho}}{\chi_{\rho}}_{\complex[G]} = \sum_{i = 1}^{n}
		m_{i}^{2}
	\end{equation}
\end{proposition}

\ifdefined\outputproof
\begin{proof}

\end{proof}
\fi

\begin{corollary}
	Soit $(G, ., 1_{G})$ un groupe fini et soit $(E,
	\rho)$ une représentation de $(G, ., 1_{G})$.

	Soit $\chi_{\rho}$ le caractère de la représentation $(E,
	\rho)$.

	Alors les assertions suivantes sont équivalentes.

	\begin{enumerate}
		\item La représentation $(E,
	\rho)$ est irréductible.
		\item Le caractère de la représentation $(E, \rho)$ est
			normé, c'est-à-dire
			\begin{equation}
				\dotprod{\chi_{\rho}}{\chi_{\rho}}_{\complex[G]} = 1
			\end{equation}
	\end{enumerate}
\end{corollary}

\ifdefined\outputproof
\begin{proof}

\end{proof}
\fi

\section{La représentation régulière}

Rappelons encore une fois que, pour un groupe fini $(G, ., 1_{G})$ l'ensemble
des fonctions $\complex[G]$ forment un espace de Hilbert de dimension
$\ordergroup{G}$.

Il est alors censé d'étudier les représentations finies $(\complex[G],
\rho)$ de $(G, ., 1_{G})$, c'est-à-dire les
représentations finies ayant comme support l'espace vectoriel
$\complex[G]$.

Nous allons nous concentrer sur une représentation, appelée
\textbf{représentation régulière}.

\begin{definition}
	Soit $(G, ., 1_{G})$ un groupe fini.
	Soit $\complex[G]$ l'espace vectoriel normé
	des fonctions de $G$ dans $\complex$ et soit $\GSsetDef{e_{g}}{g \in G}$ la
	base usuelle de $\complex[G]$.

	La représentation régulière de $(G, ., 1_{G})$ (sur $\complex$) est la
	représentation $(\complex[G], R)$ où
	\begin{equation}
		R : G \rightarrow \GSisomorphisme{\complex[G]}
	\end{equation}

	tel que, pour tout $g \in G$
	\begin{equation}
		R(g) : \complex[G] \rightarrow \complex[G] : e_{h} \rightarrow e_{gh}
	\end{equation}\footnote{On rappelle que pour une application linéaire, ce
		qui est le cas pour $R(g)$, il suffit de définir les images sur les
	éléments d'une base. Les autres éléments sont définis par linéarité.}
\end{definition}

Pour appréhender cette définition, commençons par regarder comment l'évaluation
est faite sur l'ensemble des fonctions de $\complex[G]$.

\begin{proposition}
	Soit $(G, ., 1_{G})$ un groupe fini et soit $(\complex[G],
	R)$ sa représentation régulière. Soit $(e_{g})_{g \in G}$
	la base usuelle de $\complex[G]$.

	Alors, pour tout $g \in G$, pour tout $f \in \complex[G]$ et pour tout $h
	\in G$,
	\begin{equation}
		\overbrace{R_{g}(f)}^{\in \complex[G]}(h) = \overbrace{f(g^{-1}h)}^{\in
		\complex}
	\end{equation}
\end{proposition}

\ifdefined\outputproof
\begin{proof}

\end{proof}
\fi

Calculons maintenant les représentations régulières de certains groupes.

\begin{exemple}
	Nous allons commencer par un exemple simple: le groupe à $2$ éléments
	$(\integer_{2}, +_{mod(2)}, 0)$. Représentons $\integer_{2}$ par les deux
	éléments $0$ et $1$.

	Nous devons d'abord commencer par calculer la dimension du support de la représentation
	régulière pour connaitre la taille des matrices représentant la
	représentation. Ici, le support est $\complex[\integer_{2}]$. On a vu que la
	dimension du support de la représentation régulière vaut l'ordre du groupe.
	Dans notre cas, $dim(\complex[\integer_{2}])$ car $\integer_{2}$ est d'ordre
	$2$.
	Les matrices que nous allons calculer seront des matrices complexes et
	inversibles $2 \cartprod 2$.

	Comme la représentation régulière est un morphisme, on a le neutre qui est
	envoyé sur le neutre. On a donc
	\begin{equation}
		R(0) =
		\begin{pmatrix}
			1 & 0 \\
			0 & 1
		\end{pmatrix}
	\end{equation}

	Maintenant, il nous reste à calculer $R(1)$. On a $R(1) \in
	Isom(\complex[\integer_{2}])$, c'est-à-dire une application linéaire en
	particulier. il
	nous suffit donc de calculer les images des vecteurs de bases de
	$\complex[\integer_{2}]$, qui sont $e_{0}$ et $e_{1}$.

	On a
	\begin{equation}
		R(1) (e_{0}) = e_{1 +_{mod(2)} 0} = e_{1}
	\end{equation}
	et
	\begin{equation}
		R(1) (e_{1}) = e_{1 +_{mod(2)} 1} = e_{0}
	\end{equation}

	On en déduit la matrice représentant $R(1)$:
	\begin{equation}
		R(1) =
		\begin{pmatrix}
			0 & 1 \\
			1 & 0
		\end{pmatrix}
	\end{equation}


\end{exemple}

\begin{exercice}
	\begin{enumerate}
		\item Calculer la représentation régulière de $\integer_{n}$, le groupe
			cyclique d'ordre $n$.
		\item Calculer la représentation régulière de $\integer_{2} \cartprod
			\integer_{2}$.
	\end{enumerate}
\end{exercice}

Cette représentation est assez particulière.

\begin{proposition}
	Soit $(G, ., 1_{G})$ un groupe fini.

	La représentation régulière $(\complex[G], R)$ de $(G, .,
	1_{G})$ est unitaire.
\end{proposition}

\ifdefined\outputproof
\begin{proof}
	Soit $g \in G$ et soient $f_{1}, f_{2} \in \complex[G]$. Il faut montrer que

	\begin{equation}
		\dotprod{R_{g}(f_{1})}{R_{g}(f_{2})}_{\complex[G]} =
		\dotprod{f_{1}}{f_{2}}_{\complex[G]}
	\end{equation}
\end{proof}
\fi

Qu'en est-il maintenant du caractère de la représentation régulière ?

Par définition, on a
\begin{equation}
	R_{g}(e_{h}) = e_{gh}
\end{equation}

et on a

\begin{equation}
	e_{gh} = e_{h} \Leftrightarrow g = 1_{G}
\end{equation}

Comme la trace est indépendante de la base, on peut la calculer selon la base
usuelle. Selon les remarques précédentes, on en déduit que

\begin{align}
	\chi_{R}(g) & = \delta_{1_{G}}^{g} \ordergroup{G} \\
	& = \delta_{1_{G}}^{g} dim(\complex[G]) \\
\end{align}
ou de manière plus explicite
\begin{align}
	\chi_{R}(g) & =
		\begin{cases}
			dim(\complex[G])	& \text{ si $g = 1_{G}$} \\
			0					& \text{ sinon}
		\end{cases} \\
		& =
		\begin{cases}
			\ordergroup{G}	& \text{ si $g = 1_{G}$} \\
			0					& \text{ sinon}
		\end{cases}
\end{align}

Passons à la réductibilité de la représentation régulière.

\begin{proposition}
	Soit $(G, ., 1_{G})$ un groupe fini et soit $(\complex[G],
	R)$ sa représentation régulière. Soit $(e_{g})_{g \in G}$
	la base usuelle de $\complex[G]$.

	Alors le sous-espace vectoriel de $\complex[G]$ engendré par $\sum_{g \in G} e_{g}$ est invariant pour la
	représentation régulière.

	De plus, la sous-représentation induite par le sous-espace vectoriel de
	$\complex[G]$ engendré par $\sum_{g \in G} e_{g}$ est de dimension $1$ et est équivalente à la représentation triviale.
\end{proposition}

\ifdefined\outputproof
\begin{proof}

\end{proof}
\fi

On remarque donc que la représentation régulière n'est pas irréductible si elle
n'est pas de dimension $1$.

\begin{proposition}
	Soit $(G, ., 1_{G})$ un groupe fini et soit $(\complex[G],
	R)$ sa représentation régulière. Soit $(e_{g})_{g \in G}$
	la base usuelle de $\complex[G]$.

	Soient les représentations irréductibles $(E_{1},
	\rho_{1}), \cdots, (E_{n},
	\rho_{n})$ et les naturels non nuls $m_{1}, \cdots,
	m_{n}$ tel que
	\begin{equation}
		R = \oplus_{i = 1}^{n} m_{i} \rho_{i}
	\end{equation}
	et
	\begin{equation}
		\complex[G] = \oplus_{i = 1}^{n} E_{i}
	\end{equation}.

	Alors $(E_{1}, \rho_{1}), \cdots, (E_{n},
	\rho_{n})$ sont les représentations irréductibles
	inéquivalentes de $(G, ., 1_{G})$ et $m_{i} = dim(E_{i})$.

	De plus, on a
	\begin{equation}
		\sum_{i = 1}^{n} m_{i}^{2} = \ordergroup{G}
	\end{equation}

	En d'autres termes, la décomposition de la représentation régulière d'un
	groupe fini comprend \textbf{toutes} les représentation irréductibles
	inéquivalentes du groupe fini et la dimension de ces représentations est
	donnée à travers la multiplicité.
\end{proposition}

\ifdefined\outputproof
\begin{proof}
	Commençons par la multiplicité. On sait que
	\begin{equation}
		m_{i} = \dotprod{\chi_{\rho_{i}}}{\chi_{R}}_{\complex[G]}
	\end{equation}
	En se rappelant que
	\begin{equation}
		\chi_{R}(g) = \delta_{1_{G}}^{g} \ordergroup{G}
	\end{equation}
	et
	\begin{equation}
		\chi_{\rho_{i}}(1_{G}) = dim(E_{i})
	\end{equation}
	on a
	\begin{align}
		\dotprod{\chi_{\rho_{i}}}{\chi_{R}}_{\complex[G]} & =
		\frac{1}{\ordergroup{G}} \sum_{g \in G}
		\conjuguate{\chi_{\rho_{i}}(g)} \chi_{R}(g) \\
		& = \frac{1}{\ordergroup{G}} \conjuguate{\chi_{\rho_{i}} (1_{G})}
		\chi_{R}(1_{G}) \\
		& = dim(E_{i})
	\end{align}

	Ensuite, pour obtenir l'égalité
	\begin{equation}
		\sum_{i = 1}^{n} m_{i}^{2} = \ordergroup{G}
	\end{equation}
	nous pouvons regarder le caractère de la représentation régulière. En effet,
	on a
	\begin{align}
		\ordergroup{G} & = \chi_{R}(1_{G}) \\
		& = \sum_{i = 1}^{n} m_{i} \chi_{\rho_{i}}(1_{G})
	\end{align}
	en se rappelant que
	\begin{equation}
		R = \oplus_{i = 1}^{n} m_{i} \rho_{i}
	\end{equation}
	et que le caractère d'une somme directe de représentations est la somme des
	caractères des représentations.
\end{proof}
\fi

Revenons maintenant à l'espace $\complex[G]$.

\begin{proposition}
	Soit $(G, ., 1_{G})$ un groupe fini.

	Soit $(E, \rho)$ une représentation de $(G, ., 1_{G})$ et soit $f \in
	\complex[G]$.

	Posons
	\begin{equation}
		\rho_{f} = \sum_{g \in G} f(g) \rho(g)
	\end{equation}
	En partiulier, $\rho_{f}$ est une application linéaire allant de $E$ dans
	$E$.
	Alors
	\begin{enumerate}
		\item Si $f$ est constante par classe, $\rho_{f}$ commute avec $\rho(g)$
			pour tout $g \in G$.
		\item Si $f$ est constante par classe et que $(E, \rho)$ est une
			représentation irréductible, on a
			\begin{equation}
				\rho_{f} = \frac{\ordergroup{G}
				\dotprod{\conjuguate{f}}{\chi_{\rho}}_{\complex[G]}}{\dim{(E)}}
				Id_{E}
			\end{equation}
	\end{enumerate}
	où $\conjuguate{f} : G \rightarrow \complex$ tel que
	\begin{equation}
		\conjuguate{f}(g) = \conjuguate{f(g)}.
	\end{equation}
\end{proposition}

\ifdefined\outputproof
\begin{proof}
	\begin{enumerate}
		\item On utilise $f(ghg^{-1}) = f(h)$.
		\item On utilise le lemme de Shur en utilisant le fait que $\rho_{f}$
			commute avec $\rho(g)$ pour tout $g \in G$.
			On calcule alors $\lambda$ qui donne ce qui est demandé.
	\end{enumerate}
\end{proof}
\fi

Nous en venons à un théorème important. Nous avons montré, grace au théorème
\ref{thm:uir_set_orthonormal} que l'ensemble des caractères des représentations
irréductibles inéquivalentes forment un ensemble orthonormé de $\complex[G]$.
Nous avons maintenant tous les outils pour montrer que cet ensemble est
\textbf{une base de $complex[G]$}.

\begin{theorem}
	\label{uir:uir_set_basis}
	Soit $(G, ., 1_{G})$ un groupe fini.

	Soient $(E_{1}, \rho_{1}), \cdots, (E_{N}, \rho_{N})$ les représentations
	irréductibles inéquivalentes de $(G, ., 1_{G})$ et soient $\chi_{1}, \cdots,
	\chi_{N}$ leur caractère respectif.

	Alors l'ensemble
	\begin{equation}
		\GSset{\chi_{1}, \cdots, \chi_{N}}
	\end{equation}
	forment une base orthonormé de l'espace de Hilbert $(\complex[G],
	\dotprod{.}{.}_{\complex[G]})$.
\end{theorem}

\ifdefined\outputproof
\begin{proof}
	Posons $V = \spanspace{\chi_{1}, \cdots, \chi_{N}}$.
	Pour montrer $V = \Hilbert$, nous allons montrer que $\GSortho{V} =
	\GSset{0_{\complex[G]}}$.

	Soit $f \in \GSortho{V}$, c'est-à-dire, pour tout $i \in \GSset{1, \cdots,
	N}$,
	\begin{equation}
		\dotprod{f}{\chi_{i}}_{\complex[G]} = 0
	\end{equation}
\end{proof}
\fi

On en déduit le corollaire important suivant:

% Etudier les groupes qui ont le même nombre de sous-groupes normaux ?
\begin{corollary}
	Soit $(G, ., 1_{G})$ un groupe fini.

	Soient $(H_{1}, ., 1_{G}), \cdots, (H_{M}, ., 1_{G})$ les sous-groupes normaux
	de $(G, ., 1_{G})$ et soient $(E_{1}, \rho_{1}), \cdots, (E_{N}, \rho_{N})$
	les représentations irréductibles et inéquivalentes de $(G, ., 1_{G})$.

	Alors, $N = M$.

	En d'autres termes, le nombre de représentations irréductibles
	inéquivalentes d'un groupe fini est le même que le nombre de sous-groupes
	normaux de ce groupe.
\end{corollary}

\ifdefined\outputproof
\begin{proof}

\end{proof}
\fi

Nous en venons aux relations de complétude des représentations irréductibles
inéquivalentes d'un groupe fini.

\begin{proposition}
	Soit $(G, ., 1_{G})$ un groupe fini.

	Soient $(E_{1}, \rho_{1}), \cdots, (E_{N}, \rho_{N})$ les représentations
	irréductibles inéquivalentes de $(G, ., 1_{G})$ et soient $\chi_{1}, \cdots,
	\chi_{N}$ leur caractère respectif.

	Soient $(H_{1}, ., 1_{G}), \cdots, (H_{N}, ., 1_{G})$ les sous-groupes normaux
	de $(G, ., 1_{G})$.

	Soient $g, g' \in G$.

	Alors, on a
	\begin{enumerate}
		\item Si $gH \neq g'H$, alors
			\begin{equation}
				\sum_{i = 1}^{N} \conjuguate{\chi_{i}(g)} \chi_{i}(g') = 0
			\end{equation}
		\item Sinon
			\begin{equation}
				\frac{1}{\ordergroup{G}} \sum_{i = 1}^{N}
				\conjuguate{\chi_{i}(g)} \chi_{i}(g') =
				\frac{1}{\ordergroup{H_{g}}}
			\end{equation}
			où $H_{g}$ est le sous-groupe normal contenant les éléments $g$ et
			$g'$.
	\end{enumerate}
\end{proposition}

\ifdefined\outputproof
\begin{proof}

\end{proof}
\fi

Remarquons que la deuxième équation donne
\begin{equation}
	\frac{1}{\ordergroup{G}} \sum_{i = 1}^{N} \norm{\chi_{i}(g)}^{2} =
	\frac{1}{\ordergroup{H_{g}}}
\end{equation}
