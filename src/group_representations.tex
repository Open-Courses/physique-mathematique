\chapter{Théorie des représentations des groupes}


\begin{definition}
	Soit $E$ un espace vectoriel sur un corps $K$, et soit $G$ un groupe.
	\textbf{Une représentation de $G$ dans $E$} est une action de groupe
	\begin{equation}
		\GSfunction{\sigma}{G}{GL(E)}
		\label{definition_representation}
	\end{equation}
	où $GL(E)$ est l'ensemble des automorphismes linéaires de $E$.
\end{definition}

\begin{remarque}
	--- On note aussi parfois $GL(E)$ par $Isom(E)$. La notation $GL(E)$ fait le
	lien avec l'isomorphisme qui existent en dimension finie entre l'espace
	vectoriel des matrices carrées inversibles et les automorphismes linéaires
	(et donc continues).

	--- Les notions définies dans le chapitre\ref{chapter_group_theory} sur les
	actions des groupe restent bien sûr valable.
\end{remarque}

\begin{definition}
	Une représentation $\GSfunction{\sigma}{G}{GL(E)}$ de $G$ dans $E$ est dite
	\textbf{finie} si $E$ est de dimension finie. Si $E$ n'est pas de dimension
	finie, on dit que $\sigma$ est \textbf{une représentation infinie}.
\end{definition}

