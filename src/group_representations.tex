\chapter{Théorie des représentations des groupes}

\section{Représentation : définition}
\begin{definition}
	Soit $E$ un espace vectoriel sur un corps $\mathbb{K}$, et soit $G$ un groupe.
	\textbf{Une représentation de $G$ dans $E$} est une action de groupe
	\begin{equation}
		\GSfunction{T}{G}{GL(E)}
		\label{definition_representation}
	\end{equation}
	où $GL(E)$ est l'ensemble des automorphismes linéaires de $E$.
\end{definition}

\begin{remarque}
	--- On note aussi parfois $GL(E)$ par $Isom(E)$. La notation $GL(E)$ fait le
	lien avec l'isomorphisme qui existent en dimension finie entre l'espace
	vectoriel des matrices carrées inversibles et les automorphismes linéaires
	(et donc continues).

	--- Les notions définies dans le chapitre\ref{chapter_group_theory} sur les
	actions des groupe restent bien sûr valables.

	--- L'image de $T$, noté $T(G)$ est \textbf{une représentation de $T$}.
	C'est également \textbf{un sous-groupe} de $GL(E)$ !

	--- Soit $g \in G$. On note souvent $T(g) = T_{g} \in GL(E)$ pour éviter
	d'utiliser trop de parenthèses.
\end{remarque}

\begin{definition}
	Une représentation $\GSfunction{T}{G}{GL(E)}$ de $G$ dans $E$ est dite
	\textbf{finie} si $E$ est de dimension finie. Si $E$ n'est pas de dimension
	finie, on dit que $T$ est \textbf{une représentation infinie}.
\end{definition}

En physique, on s'intéresse particulièrement aux représentations finies, et le
corps de base est $\complex$. La plupart des propositions et théorèmes sont
valables quelque soit le corps, et quelque soit la dimension.
Lorsque nous sommes en face de cas particuliers, la distinction sera faite.

On notera par $\GSrepr{G}{E}$ l'ensemble des représentations de $G$ dans $E$,
$\GSreprf{G}{E}$ les représentations finies, et $\GSrepri{G}{E}$ les
représentations infinies.

Prenons $T \in \GSreprf{G}{E}$, et notons $(\ket{e_{1}}, \cdots, \ket{e_{n}})$
une base de $E$ ($E$ est de dimension finie car $T$ est une représentation
finie).

Pour $g \in G$, on a $T(g)$ qui peut être représentée par une matrice $n x n$.
Cette matrice sera notée $D(g)$.

On a alors
\begin{equation}
	T(g)(\ket{e_{i}}) = D(g)^{j}_{i} \ket{e_{j}}
	\footnote{On utilise la convention d'Einstein}
\end{equation}


\begin{proposition}
	Soit $T \in \GSrepr{G}{E}$.
	Alors les assertions suivantes sont équivalentes:
	\begin{enumerate}
		\item $T$ est dégénérée.
		\item il existe un sous groupe normal non trivial $H$ de $G$ tel que
			toute représentation $S \in \GSrepr{G/H}{E}$ est dégénérée.
	\end{enumerate}
\end{proposition}

\begin{proof}

\end{proof}

\begin{corollary}
	Soit $G$ un groupe simple. Alors toute représentation $T \in \GSrepr{G}{E}$
	est fidèle.
\end{corollary}

\begin{proof}

\end{proof}

\section{Sous-espace invariant et représentation irréductible}

\begin{definition} [Sous espace invariant]
	Soit $G$ un groupe, $E$ un espace vectoriel sur $K$, et $V$ un sous-espace
	vectoriel sur $K$.

	Soit $T \in \GSrepr{G}{E}$

	On dit que $V$ est \textbf{invariant pour $T$} si
	\begin{equation}
		\forall g \in G, \, T_{g}(V) \subseteq V
		\label{definition_invariant_subspace}
	\end{equation}
\end{definition}

\begin{proposition}
	$\left\{ O_{E} \right\}$ et $E$ sont des espaces invariants. On les appelle
	les espaces invariants \textbf{triviaux}.
\end{proposition}

\begin{proof}
	
\end{proof}

\begin{definition} [Représentation irréductible]
	Soit $T \in \GSrepr{G}{E}$. On dit que $T$ est \textbf{irréductible} s'il
	n'admet que des sous-espaces invariant triviaux. S'il existe un sous-espace
	invariant non-trivial, $T$ est dit \textbf{réductible}.
\end{definition}

\section{Représentations équivalentes}

\begin{definition}
	Soit $S \in \GSrepr{G}{E}$ et $T \in \GSrepr{G}{F}$.
	On dit que \textbf{$S$ et $T$} sont équivalentes s'il existe un isomorphisme
	$A$ de $E$ dans $F$ ($A \in Isom(E, F)$) tel que:
	\begin{equation}
		\forall g \in G, \, A \circ S(g) = T(g) \circ A.
	\end{equation}
	De manière équivalente, on a $A \circ S(g) \circ A^{-1} = T(g)$.
	
	On peut également dire que $T$ est un conjugué de $S$. On peut donc dire que
	les représentations équivalentes d'une représentation donnée sont ses
	conjugués.
\end{definition}

De cette manière, on \textit{classe} les représentations de $G$ car la
relation 'être équivalentes' pour deux représentations est une relation
d'équivalence. On parlera de \textbf{classe de représentation}.
Dans le suite, nous nous intéressons surtout au cas où $E = F$.

\begin{definition}
	Soit $T \in \GSreprf{G}{E}$ une représention \textbf{finie}.
	On définit le caractère de $T$ comme la fonction:
	\begin{equation}
		\GSfunction{\chi}{G}{\mathbb{K}} : g \rightarrow Tr(T_{g})
	\end{equation}

	où $\GSfunction{Tr}{GL(E)}{\complex}$ est la trace.
\end{definition}

\begin{proposition}
	Soit $S, T$ deux représentations finies de $G$ équivalentes. Alors $S$ et
	$T$ ont le même caractère. C'est-à-dire que la trace est stable dans les
	classes de représentation.
\end{proposition}

\begin{proof}
	
\end{proof}


\section{Représentation dans les espaces de Hilbert}

Prenons maintenant le cas où $T \in \GSrepr{G}{\Hilbert}$ où $\Hilbert$ est un
espace de Hilbert.
On peut alors définir les opérateurs unitaires qui sont les opérateurs $U$ tel
que $U^{*}$, l'opérateur adjoint, est égale à $U^{-1}$.

Les représentations unitaires jouent un rôle important. En effet, ce sont des
isométries

\begin{definition}
	Soit $T \in \GSrepr{G}{\Hilbert}$. On dit que $T$ est \textbf{une
	représentation unitaire} si pour tout élément $g$ de $G$, $T(g)$ est unitaire.
\end{definition}

Rappelons que dans un espace de Hilbert, on peut définir l'orthogonal d'un
sous-espace vectoriel. Etant donné $V$ un sous-espace vectoriel de $\Hilbert$,
on définit $\GSortho{V}$. On a alors $V$ et $\GSortho{V}$ qui sont en somme
directe ie $\Hilbert = V \oplus \GSortho{V}$. On dit que $\GSortho{V}$ est le
complémentaire orthogonal.

On va alors définir un nouveau de type de représentation:


\begin{definition}
	Soit $T \in \GSreprf{G}{\Hilbert}$ et soit $V$ un sous-espace invariant de
	$T$.
	On dit que $T$ est \textbf{complètement réductible ou décomposable} si
	$\GSortho{V}$ est également invariant.
\end{definition}

Le nom de décomposable se rapporte à la représentation matricielle de $T$. En
effet, dans ce cas $D(g)$ est décomposable en deux sous-matrices.

%Ajouter la représentation matricielle.

Donnons quelques représentation décomposables.

\begin{proposition}
	Soit $T \in \GSreprf{G}{\Hilbert}$ qui est réductible. Alors $T$ est
	complètement réductible.
\end{proposition}

\begin{proof}
	Comme $T$ est réductible, il existe un sous-espace invariant non trivial.
	Notons le $V$.
	Nous voulons alors montrer que $\GSortho{V}$ est également invariant pour
	$T$. On aura alors que $T$ est complètement réductible.
\end{proof}

\begin{proposition}
	Soit $T \in \GSreprf{G}{\Hilbert}$. Alors $T$ est équivalente à une
	représentation unitaire.
\end{proposition}

\begin{proof}
	
\end{proof}

Cette proposition nous informe sur les classes des représentations \textit{d'un groupe
fini dans un espace de Hilbert}. Nous pouvons toujours nous concentrer sur les
représentations unitaires.

\section{Représentation de somme directe}

\begin{definition}
	Soit $V$ et $\GSortho{V}$ son complémentaire orthogonal dans $\Hilbert$.
	Supposons qu'on ait une représentation $T \in \GSreprf{G}{\Hilbert}$ tel que
	$V$ et $\GSortho{V}$ sont invariants sur $T$.

	Posons $T_{1}(g) = T(g)_{|V}$ et $T_{2}(g) = T(g)_{|\GSortho{V}}$
	On dit alors que $T$ est \textbf{somme directe} de $T_{1}$ et $T_{2}$ et on
	note $T = T_{1} \oplus T_{2}$.
\end{definition}

\section{Lemme de Schur}

\begin{proposition} [Lemme de Shur 1]
	\label{shur_lemma_1}
	Soit $T \in \GSrepr{G}{\Hilbert}$ irréductible.
	Soit $\GSfunction{A}{\Hilbert}{\Hilbert}$
	tel que $A \in \GScontinueEndo{\Hilbert}$ et

	\begin{equation}
		\forall g \in G, \, A \circ T(g) = T(g) \circ A.
	\end{equation}

	Alors $A = \lambda Id_{\Hilbert}$.

	En d'autre terme, deux représentations équivalentes différent d'une
	translation.
	% A reformuler et donner une interprétation.
\end{proposition}

\section{Orthonormalité et relations de complétude des matrices de
représentations irréductibles}

Nous nous plaçons toujours dans l'étude des représentation dans un espace de
Hilbert $\Hilbert$.

