\chapter{Théorie des représentations des groupes}

\section{Définitions et premières propriétés}

\subsection{Représentations}

\begin{definition}
	Soit $G$ un groupe.

	\textbf{Une représentation de $G$} est un couple $(T, E)$ où $E$
	est un espace vectoriel normé et $T$
	est une action de groupe

	\begin{equation}
		\GSfunction{T}{G}{GL(E)}
		\label{definition_representation}
	\end{equation}

	où $GL(E)$ est l'ensemble des automorphismes linéaires de $E$.

	Quand $E$ est donné, on parle de \textbf{représentation de $G$ dans $E$}.
\end{definition}

\begin{remarque}
	\begin{itemize}
		\item On note aussi parfois $GL(E)$ par $Isom(E)$. La notation $GL(E)$ fait le
	lien avec l'isomorphisme qui existent en dimension finie entre l'espace
	vectoriel des matrices carrées inversibles et les automorphismes linéaires
	(et continues car dimension finie).

		\item Les notions définies dans le chapitre \ref{chapter_group_theory} sur les
	actions des groupe restent bien sûr valables.

		\item L'image de $T$, noté $T(G)$ est \textbf{une réalisation de $G$ dans $E$}.
	C'est également \textbf{un sous-groupe} de $GL(E)$ ! On parlera aussi de
	représentation de $G$ de $T$ tout simplement.

		\item Soit $g \in G$. On note souvent $T(g) = T_{g} \in GL(E)$ pour éviter
	d'utiliser trop de parenthèses.
	\end{itemize}
\end{remarque}

\begin{definition}
	Une représentation $\GSfunction{T}{G}{GL(E)}$ de $G$ dans $E$ est dite
	\textbf{finie} si $E$ est de dimension finie. Si $E$ n'est pas de dimension
	finie, on dit que $T$ est \textbf{une représentation infinie}.

	La dimension de $T(G)$ est appelé \textbf{le degré de la représentation}.
\end{definition}

En physique, on s'intéresse particulièrement aux représentations finies, et le
corps de base est $\complex$. La plupart des propositions et théorèmes sont
valables quelque soit le corps, et quelque soit la dimension.
Lorsque nous sommes en face de cas particuliers, la distinction sera faite.

On notera par $\GSrepr{G}{E}$ l'ensemble des représentations de $G$ dans $E$,
$\GSreprf{G}{E}$ les représentations finies, et $\GSrepri{G}{E}$ les
représentations infinies.

Prenons $T \in \GSreprf{G}{E}$, et notons $(\ket{e_{1}}, \cdots, \ket{e_{n}})$
une base de $E$ ($E$ est de dimension finie car $T$ est une représentation
finie).

Pour $g \in G$, on a $T(g)$ qui peut être représentée par une matrice $n
\cartprod n$.
Cette matrice sera notée $D(g)$.

On a alors
\begin{equation}
	T(g)(\ket{e_{i}}) = D(g)^{j}_{i} \ket{e_{j}}
	\footnote{On utilise la convention d'Einstein}
\end{equation}

Prenons comme exemple $\mathbb{K} = \complex$, et prenons une représentation $T
\in \GSreprf{G}{\complex^{*}}$ où $G$ est un groupe fini.
Alors $T(g)^{n} = 1_{\complex}$, donc l'image de $T$ est les racines de l'unité.
On appelle $T$ \textbf{la représentation unité}.

%Prenons maintenant un groupe $G$ d'ordre fini $g$, et un espace vectoriel $E$ de
%dimension $g$. Soit $(e_{1}, \cdots, e_{g})$ une base de $E$. Numérotons les
%élements $g$ de $G$ par des indices $i$. On peut alors créer l'application:

%\begin{align}
	%\GSfunction{T}{G}{GL(E)} : g_{i} \rightarrow
	%T_{e_{i}} 	& : E \rightarrow E \\
				%& : e_{j} \rightarrow e_{i + j}
%\end{align}
%Cette représentation est appelée \textbf{la représentation régulière de $G$}.

\begin{proposition}
	Soit $T \in \GSrepr{G}{E}$.
	Alors les assertions suivantes sont équivalentes:
	\begin{enumerate}
		\item $T$ est dégénérée.
		\item il existe un sous groupe normal non trivial $H$ de $G$ tel que
			toute représentation $S \in \GSrepr{G/H}{E}$ est dégénérée.
	\end{enumerate}
\end{proposition}

\ifdefined\outputproof
\begin{proof}

\end{proof}
\fi

\begin{corollary}
	Soit $G$ un groupe simple. Alors toute représentation $T \in \GSrepr{G}{E}$
	non nulle est fidèle.
\end{corollary}

\ifdefined\outputproof
\begin{proof}
	Si $T$ n'est pas fidèle, alors $G$ possède un noyau non vide. Comme $T$ est
	non nulle, alors $\ker{T} \neq G$, et donc il existe un sous-groupe normal
	non trivial. Or, $G$ est simple.
\end{proof}
\fi

\subsection{Sous-espace invariant et représentation irréductible}

\begin{definition} [Sous espace invariant]
	Soient $G$ un groupe, $E$ un espace vectoriel sur $\mathbb{K}$, et $V$ un sous-espace
	vectoriel de $E$.

	Soit $T \in \GSrepr{G}{E}$

	On dit que $V$ est \textbf{invariant pour $T$} si
	\begin{equation}
		\forall g \in G, \, T_{g}(V) \subseteq V
		\label{definition_invariant_subspace}
	\end{equation}
\end{definition}

\begin{proposition}
	$\left\{ O_{E} \right\}$ et $E$ sont des espaces invariants. On les appelle
	les espaces invariants \textbf{triviaux}.
\end{proposition}

\ifdefined\outputproof
\begin{proof}

\end{proof}
\fi

\begin{proposition}
	Soit $T \in \GSrepr{G}{E}$. Soit $V$ un sous-espace vectoriel de $E$
	invariant pour $T$.

	Alors la fonction:

	\begin{equation}
		T^{V} : G \rightarrow GL(V) : g \rightarrow T(g)_{|V}
	\end{equation}

	est une représentation de $G$ dans $V$.
\end{proposition}

\begin{definition}
	Reprenons les mêmes notations qu'à la proposition précédente.
	On dit alors que $T^{V}$ est \textbf{une sous-représentation de $T$}.
\end{definition}

\begin{proposition}
	Soit $T \in \GSrepr{G}{E}$. Soit $W$ un sous-espace vectoriel de $E$
	invariant sous $T$. Alors il existe un supplémentaire de $W$, noté $W'$,
	invariant sous $T$.
\end{proposition}

En particulier, on a donc $E = W \oplus W'$, et l'étude de la représentation $T$
se résume à étudier la représentation $T^{W}$ et $T^{W'}$. On note $T = T^{W}
\oplus T^{W'}$ et on dit que $T$ est somme directe de $T^{W}$ et $T^{W'}$.

Nous pouvons alors itérer le processus sur $T^{W}$ et $T^{W'}$ en cherchant des
espaces invariants, jusqu'à ne plus trouver que des espaces invariants triviaux.

Cela nous amène à définir la notion de représentation irréductible.

\begin{definition} [Représentation irréductible]
	Soit $T \in \GSrepr{G}{E}$. On dit que $T$ est \textbf{irréductible} s'il
	n'admet que des sous-espaces invariant triviaux. S'il existe un sous-espace
	invariant non-trivial, $T$ est dit \textbf{réductible}.
\end{definition}

\begin{exercice}
	Soit $T$ une rerésentation d'un groupe $G$ de degré 1. Alors $T$ est
	irréductible.
\end{exercice}

Précédemment, nous avons écrit $T$ en somme directe de deux autres
représentations $T^{W}$ et $T^{W'}$. Ces deux dernières n'étaient pas nécessairement irréductible,
mais nous pouvons réitérer le processus sur $T^{W}$ et $T^{W'}$ en étudiant les
espaces invariants.

Nous obtenons alors un théorème intéressant:

\begin{proposition}
	\label{prop:decomposition_somme_directe}
	Soit $T \in \GSrepr{G}{E}$. Alors $T$ est somme directe de représentation
	irréductible.
\end{proposition}

La proposition \ref{prop:decomposition_somme_directe} nous informe qu'étudier
les représentations irréductibles suffit à étudier toutes les représentations.
Il suffira de \textit{réduire} notre représentation.

\subsection{Représentations équivalentes}

La définition suivante peut être généralisée à deux espaces vectoriels
différents $E$ et $F$, et pas seulement à un même espace $E$.

\begin{definition}
	Soit $S \in \GSrepr{G}{E}$ et $T \in \GSrepr{G}{E}$.
	On dit que \textbf{$S$ et $T$ sont équivalentes} s'il existe un isomorphisme
	linéaire continue $A$ de $E$ dans $E$ ($A \in Isom(E)$) tel que:
	\begin{equation}
		\forall g \in G, \, A \circ S(g) = T(g) \circ A.
	\end{equation}
	De manière équivalente, on a $A \circ S(g) \circ A^{-1} = T(g)$.

	On peut également dire que $T$ est un conjugué de $S$. On peut donc dire que
	les représentations équivalentes d'une représentation donnée sont ses
	conjugués.
\end{definition}

De cette manière, on \textit{classe} les représentations de $G$ car la
relation 'être équivalentes' pour deux représentations est une relation
d'équivalence. On parlera de \textbf{classe de représentation}.


\subsection{Caractère d'une représentation finie}

\begin{definition}
	Soit $T \in \GSreprf{G}{E}$ une représention \textbf{finie}.
	On définit le caractère de $T$ comme la fonction:
	\begin{equation}
		\GSfunction{\chi}{G}{\mathbb{K}} : g \rightarrow Tr(T_{g})
	\end{equation}

	où $\GSfunction{Tr}{GL(E)}{\mathbb{K}}$ est la trace.
\end{definition}

On a pour chaque représentation que $\chi(1_{G}) = n$ où $n$ est la dimension de
$E$.
On a également que $\chi(h g h^{-1}) = \chi(g)$.

Dans le cas où $\mathbb{K} = \complex$, on obtient la proposition suivante, laissée en exercice.

\begin{proposition}
	$\chi(s^{-1}) = \conjuguate{\chi(s)}$
\end{proposition}

En particulier, $\chi$ est constante ssi $T$ est la représentation triviale (ie $\forall g \in G$, $T(g) =
Id_{E}$).

\begin{proposition}
	Soit $S, T$ deux représentations finies de $G$ équivalentes. Alors $S$ et
	$T$ ont le même caractère. C'est-à-dire que la trace est stable dans les
	classes de représentation.
\end{proposition}

\ifdefined\outputproof
\begin{proof}

\end{proof}
\fi

Prenons maintenant un espace vectoriel $E$, et deux sous espaces vectoriels
$V_{1}$ et $V_{2}$ qui sont en somme directe.
Soient $T_{1} \in \GSreprf{G}{V_{1}}$, et $T_{2} \in \GSreprf{G}{V_{2}}$, et
$\chi_{1}$, $\chi_{2}$ leur caractère respectif.

Construisons $T : G \rightarrow GL(V_{1} \oplus V_{2})$ tel que $T(g) = T_{1}(g)
+ T_{2}(g)$ (l'addition est vue dans $GL(E))$, et posons $\chi$ son caractère.

Alors, on montre que $\chi = \chi_{1} + \chi_{2}$.


\subsection{Sous groupe commutatif}

Nous avons remarqué, grace à \ref{prop:decomposition_somme_directe}, que le
simple fait d'étudier les représentations irréductibles nous permettait de
connaitre toutes les représentations.

La proposition suivante va nous permettre de donner un critère pour
éliminer certaines représentations qui ne sont pas irréductibles.

\begin{proposition}
	Soient $G$ un groupe, et $H$ un sous groupe abélien.
	Soit $T \in \GSrepr{G}{V}$.
	Alors $V$ est de dimension inférieure ou égale à
	$\frac{\ordergroup{G}}{\ordergroup{H}}$.
\end{proposition}

\section{Représentation dans les espaces de Hilbert}

Nous supposerons maintenant que nous sommes dans le cas $\mathbb{K} = \complex$
ou $\mathbb{K} = \real$. Une généralisation est à envisager.

Depuis le début, nous nous sommes uniquement intéressés au cas où $E$ est un
espace vectoriel normé quelconque.

Prenons maintenant le cas où $T \in \GSrepr{G}{\Hilbert}$ où $\Hilbert$ est un
espace de Hilbert.
On peut alors définir les opérateurs unitaires qui sont les opérateurs $U$ tel
que $U^{*}$, l'opérateur adjoint, est égale à $U^{-1}$.

Les représentations unitaires jouent un rôle important. En effet, ce sont des
isométries qui sont les isomorphismes gardant en plus les distances dans les
espaces de Banach. Les isométries transportent également l'information du
produit scalaire. Les opérateurs unitaires conservent les symétries des
problèmes car ils ont la propriété de conserver les angles, les distances et le
produit scalaire.

\begin{definition}
	Soit $T \in \GSrepr{G}{\Hilbert}$. On dit que $T$ est \textbf{une
	représentation unitaire} si pour tout élément $g$ de $G$, $T(g)$ est unitaire.
\end{definition}

Rappelons que dans un espace de Hilbert, on peut définir l'orthogonal d'un
sous-espace vectoriel. Etant donné $V$ un sous-espace vectoriel fermé de $\Hilbert$,
on définit $\GSortho{V}$ comme l'ensemble des vecteurs perpandiculaires avec
tous les vecteurs $V$. On a alors $V$ et $\GSortho{V}$ qui sont en somme
directe ie $\Hilbert = V \oplus \GSortho{V}$. On dit que $\GSortho{V}$ est le
complémentaire orthogonal.

On va alors définir un nouveau de type de représentation:

\begin{definition}
	Soit $T \in \GSrepr{G}{\Hilbert}$ et soit $V$ un sous-espace invariant de
	$T$.
	On dit que $T$ est \textbf{complètement réductible ou décomposable} si
	$\GSortho{V}$ est également invariant.
\end{definition}

Le nom de décomposable se rapporte à la représentation matricielle de $T$. En
effet, dans ce cas $D(g)$ est décomposable en deux sous-matrices.

%Ajouter la représentation matricielle.

Donnons quelques représentation décomposables.

\begin{proposition}
	Soit $T \in \GSrepr{G}{\Hilbert}$ unitaire qui est réductible. Alors $T$
	est complètement réductible.
\end{proposition}

\ifdefined\outputproof
\begin{proof}
	Comme $T$ est réductible, il existe un sous-espace invariant non trivial.
	Notons le $V$.
	Nous voulons alors montrer que $\GSortho{V}$ est également invariant pour
	$T$. On aura alors que $T$ est complètement réductible.
\end{proof}
\fi

\begin{proposition}
	Soit $T \in \GSrepr{G}{\Hilbert}$ où $G$ est fini. Alors $T$ est équivalente à une
	représentation unitaire.
\end{proposition}

\ifdefined\outputproof
\begin{proof}

\end{proof}
\fi

Cette proposition nous informe sur les classes des représentations \textit{d'un groupe
fini dans un espace de Hilbert}. Nous pouvons toujours nous concentrer sur les
représentations unitaires.

\subsection{Lemme de Schur}

\begin{proposition} [Lemme de Shur 1]
	\label{lemma:shur_lemma_1}
	Soit $T \in \GSrepr{G}{\Hilbert}$ irréductible.
	Soit $\GSfunction{A}{\Hilbert}{\Hilbert}$
	tel que $A \in \GScontinueEndo{\Hilbert}$ et

	\begin{equation}
		\forall g \in G, \, A \circ T(g) = T(g) \circ A.
	\end{equation}

	Alors $A = \lambda Id_{\Hilbert}$ ou $A = 0_{\Hilbert}$.

	En d'autres termes, si un opérateur $A : \Hilbert \rightarrow \Hilbert$
	commute avec tous les opérateurs $T(g)$, $g$ parcourant $G$, alors il est
	obligé que $A$ soit un opérateur de translation.
	% A reformuler et donner une interprétation.
\end{proposition}

Nous avons une conséquence direct du \ref{lemma:shur_lemma_1}.

\begin{corollary}
	Soit $T \in \GSreprf{G}{\Hilbert}$ une représentation irréductible où $G$
	est un groupe commutatif. Alors $\Hilbert$ est de dimension $1$, et donc
	$\Hilbert = \real$ ou $\Hilbert = \complex$.
\end{corollary}

Ce corollaire nous montre que les seules représentations irréductibles d'un
groupe commutatif sont de degré 1. C'est-à-dire que lorsque nous étudions les
représentations irréductibles d'un groupe abélien, nous étudions des
sous-groupes (finis) de $\real$ ou $\complex$.

En particulier, les seuls sous-groupes finis de $\complex$ sont les groupes des
racines de l'unité. Étudier les représentation irréductibles de $G$ sur
$\complex$ revient à étudier les sous-groupes abéliens de $G$, et donc en
particulier les sous-groupes cycliques.

\begin{proposition} [Lemme de Shur 2]
	\label{lemma:shur_lemma_2}
	Soient $\Hilbert$ et $\mathcal{K}$ deux espaces de Hilbert, et $G$ un groupe.
	Soient $S \in \GSrepr{G}{\Hilbert}$ et $T \in \GSrepr{G}{\mathcal{K}}$ deux
	représentations de $G$.

	Soit $A \in \GScontinueHomo{\Hilbert}{\mathcal{K}}$ tel que:

	\begin{equation*}
		\forall g \in G, A \circ S(g) = T(g) \circ A
	\end{equation*}

	Alors, soit $A = O_{\Hilbert, \mathcal{K}}$, soit $A = \lambda Id_{\Hilbert,
		\mathcal{K}}$.
\end{proposition}

\ifdefined\outputproof
\begin{proof}
	Remarquons que dans le deuxième cas, nous avons comme conséquences que
	$\Hilbert$ est isomorphe à $\mathcal{K}$, et que $S$ et $T$ sont
	équivalentes.
\end{proof}
\fi

\subsection{Orthonormalité et relations de complétude des matrices de
représentations irréductibles}

Nous nous plaçons toujours dans l'étude des représentation dans un espace de
Hilbert $\Hilbert$.



