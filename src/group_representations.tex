\chapter{Théorie des représentations des groupes}


\begin{definition}
	Soit $E$ un espace vectoriel sur un corps $\mathbb{K}$, et soit $G$ un groupe.
	\textbf{Une représentation de $G$ dans $E$} est une action de groupe
	\begin{equation}
		\GSfunction{T}{G}{GL(E)}
		\label{definition_representation}
	\end{equation}
	où $GL(E)$ est l'ensemble des automorphismes linéaires de $E$.
\end{definition}

\begin{remarque}
	--- On note aussi parfois $GL(E)$ par $Isom(E)$. La notation $GL(E)$ fait le
	lien avec l'isomorphisme qui existent en dimension finie entre l'espace
	vectoriel des matrices carrées inversibles et les automorphismes linéaires
	(et donc continues).

	--- Les notions définies dans le chapitre\ref{chapter_group_theory} sur les
	actions des groupe restent bien sûr valables.

	--- L'image de $T$, noté $T(G)$ est \textbf{une représentation de $T$}.

	--- Soit $g \in G$. On note souvent $T(g) = T_{g} \in GL(E)$ pour éviter
	d'utiliser trop de parenthèses.

\end{remarque}

\begin{definition}
	Une représentation $\GSfunction{T}{G}{GL(E)}$ de $G$ dans $E$ est dite
	\textbf{finie} si $E$ est de dimension finie. Si $E$ n'est pas de dimension
	finie, on dit que $T$ est \textbf{une représentation infinie}.
\end{definition}

En physique, on s'intéresse particulièrement aux représentations finies, et le
corps de base est $\complex$. La plupart des propositions et théorèmes sont
valables quelque soit le corps, et quelque soit la dimension.
Lorsque nous sommes en face de cas particuliers, la distinction sera faite.

On notera par $\GSrepr{G}{E}$ l'ensemble des représentations de $G$ dans $E$,
$\GSreprf{G}{E}$ les représentations finies, et $\GSrepri{G}{E}$ les
représentations infinies.

\begin{definition}
	Soit $G$ un groupe, $E$ un espace vectoriel sur $K$, et $V$ un sous-espace
	vectoriel sur $K$.

	Soit $T \in \GSrepr{G}{E}$

	On dit que $V$ est \textbf{invariant pour $T$} si
	\begin{equation}
		\forall g \in G, \, T_{g}(V) \subseteq V
		\label{definition_invariant_subspace}
	\end{equation}
\end{definition}

\begin{proposition}
	Soit $T \in \GSrepr{G}{E}$.
	Alors les assertions suivantes sont équivalentes:
	\begin{enumerate}
		\item $T$ est dégénérée.
		\item il existe un sous groupe normal non trivial $H$ de $G$ tel que
			toute représentation $S \in \GSrepr{G/H}{E}$ est dégénérée.
	\end{enumerate}
\end{proposition}

\begin{proof}

\end{proof}

\begin{corollary}
	Soit $G$ un groupe simple. Alors toute représentation $T \in \GSrepr{G}{E}$
	est fidèle.
\end{corollary}

\begin{definition}
	Soit $S \in \GSrepr{G}{E}$ et $T \in \GSrepr{G}{F}$.
	On dit que \textbf{$S$ et $T$} sont équivalentes s'il existe un isomorphisme
	$A$ de $E$ dans $F$ tel que:
	\begin{equation}
		\forall g \in G, \, A \circ S(g) = T(g) \circ A.
	\end{equation}
	De manière équivalente, on a $A \circ S(g) \circ A^{-1} = T(g)$.
\end{definition}

\begin{definition}
	Soit $T \in \GSreprf{G}{E}$ une représention \textbf{finie}.
	On définit le caractère de $T$ comme la fonction:
	\begin{equation}
		\GSfunction{\chi}{G}{\mathbb{K}} : g \rightarrow Tr(T_{g})
	\end{equation}
\end{definition}

\begin{proposition}
	Soit $S, T$ deux représentations finies de $G$ équivalentes. Alors $S$ et
	$T$ ont le même caractère.
\end{proposition}
