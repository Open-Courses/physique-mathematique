\chapter{Théorie des groupes}
\label{chapter_group_theory}

\section{Définitions}

\begin{definition}
	Soit $G$ un groupe, et $H$ un sous groupe de $G$.
	$H$ est \textbf{un sous groupe normal} ou \textbf{un sous groupe invariant}
	si:
	\begin{equation}
		\forall g \in G, gH = Hg.
		\label{definition_normal_subgroup}
	\end{equation}

	De manière équivalente, $H$ est un sous groupe normal si $H$ est invariant
	par conjugaison (d'où le nom de sous groupe invariant), c'est-à-dire:
	\begin{equation}
		\forall g \in G, gHg^{-1} = H.
		\label{definition_normal_subgroup2}
	\end{equation}
\end{definition}

\begin{definition} [Action de groupe]
	Soit $G$ un groupe et $E$ un ensemble quelconque (sans structure
	algébrique). Soit $S_{E}$ l'ensemble des permutations sur $E$.

	Soit
	\begin{equation}
		\GSfunction{\sigma}{G}{S_{E}}
		\label{definition_group_action}
	\end{equation}

	$\sigma$ est \textbf{une action de $G$ dans $S_{E}$} si $\sigma$ est un
	morphisme de groupe entre $G$ et $S_{E}$.
\end{definition}

\begin{definition} [Action fidèle]
	Soit $\sigma$ une action de groupe de $G$ dans $S_{E}$.
	$\sigma$ est \textbf{fidèle} si $\sigma$ est injective, ie $\ker{\sigma} =
	\left\{ 0_{G} \right\}$
\end{definition}
